\begin{otherlanguage*}{english}
%\part{Abstract}
\begin{center}
    {\normalfont \textbf{
    	ABSTRACT\\
    	Semenov A.K. Electrophysical properties of multiphase disperse systems.} -- Qualification scientific paper, manuscript.}
\end{center}
\vskip 5pt

Thesis for a Candidate of Science in Physics and Mathematics (Philosophy Doctor) degree by specialty 01.04.02 -- theoretical physics. -- Odesa I.I.~Mechnikov National University, the MES of Ukraine, Odesa, Ukraine, \the\year.

\vskip 5pt

Thesis is concentrated on construction of a theoretical model for effective quasistatic electrophysical characteristics of disordered three-dimensional macroscopically homogeneous and isotropic systems of spherical particles with hard-core--penetrable-shell morphology, its analysis and applications to description of quasistatic electric conductivity $ \sigma_{\rm eff}$ and dielectric permittivity $\varepsilon_ {\rm eff} $ of multiphase heterogeneous systems.
These systems are one of the most common, but the least theoretically studied due to the need, on the one hand, to take into account various interphase and matrix effects that form the microstructure of the system and, on the other hand, to solve essentially many-particle electrodynamic problem.
In the present work it is shown that their effective description is possible by using a model system of spherical particles with hard-core--penetrable-shell morphology. The shells in the general have a radial distribution of complex dielectric permittivity.
The local value of the complex permeability at the points of overlap of the system's components is determined by the distance to the nearest particle.
To build the model, the method of compact groups of inhomogeneities is used, which is generalized to systems with conducting components. In terms of this method, the system is considered as a set of macroscopic regions (compact groups) that are point-like with respect to the testing field, but large enough to have the properties of the whole system.
Compact groups are represented by singular contributions to the propagator, which are dominant in the quasistatic limit, which allows one to sum up the entire iterative series without the need to calculate its individual terms.
The calculation scheme is closed by the requirement to fulfill  within its framework the known boundary conditions for normal components of the complex electric current.
The final results are the integral relation which links the effective quasistatic complex permittivity of the system with complex permittivities and volume concentrations of its components and, under certain conditions, similar relations for $ \sigma_ {\rm eff} $ and $ \varepsilon_ {\rm eff} $.

To test the theoretical results, the existing data for numerical simulations within the Random Resistor Network algorithm for static conductivity of these model systems with different core diameters and thicknesses of electrically homogeneous and inhomogeneous shells were used. These data were obtained for the case when the conductivity of the shells $ \sigma_2 $ is much higher than the conductivity of the matrix $ \sigma_0 $ and of the cores $ \sigma_1 $, which is typical for composite solid (CSE) and composite polymer  (CPE) electrolytes.
Taking into account the peculiarities of the algorithm used in the simulations and the related problems of mapping the model's results on the simulation data, it is shown that the theory is able to fully reproduce these data.

It is demonstrated that the model is applicable for describing the concentration dependences of the electric conductivity $ \sigma_ {\rm eff} $ of real  composite solid and composite  polymer electrolytes.
The known experimental data of Liang for CSE formed by dispersion of  $ \rm Al_2O_3 $ particles into the polycrystalline matrix $ \rm LiI $ and of the Wieczorek group for CPEs based on poly(ethylene) oxide ($ \rm PEO $) and oxymethylene-linked  $ \rm PEO $ ($ \rm OMPEO $) with addition of $ \rm NaI $ or $ \rm LiClO_4 $ salts, with $ \rm Na_ {3.2} Zr_2P_ {0.8} Si_ { 2.2} O_ {12} $ (NASICON), $ \rm \theta-Al_2O_3 $ particles  or polyacrylamide ($ \rm PAAM $) globules as fillers,  were used for the analysis.
Data processing was performed by using model conductivity profiles $ \sigma_2 (r) $ of penetrable shells, the shape of which gradually became more complicated from a step with a constant height to sigmoid superposition.
The assumption is made and argued that the profiles $ \sigma_2 (r) $ obtained as a result of such processing can be used to analyze the role of different physicochemical mechanisms in formation of the effective conductivity $ \sigma_ {\rm eff} $.

In particular, for $ \rm LiI-Al_2O_3 $ CSE $ \sigma_2 (r) $ has two clearly defined regions. The outer region incorporates the contribution of matrix processes to the formation of $ \sigma _ {\rm eff} $. These can be uncontrolled doping of the matrix during the  experimental samples preparation, accumulation of dislocations etc.. The inner region indicates the existence of a highly conductive space charge layer around the $ \rm \theta-Al_2O_3 $ particles. It can be caused by accumulation of point defects, as evidenced by the results of comparison the obtained characteristics of this region with estimates of other authors.

The results for CPE show the presence of two or three distinct regions in $ \sigma_2 (r) $, which can be interpreted as follows. The central region in $ \sigma_2 (r) $ stands for the formation around the particles in the CPE of amorphous polymer regions with relatively high conductivity as a result of increased mobility of dissolved salt ions in these regions. The region closest to the core incorporates the total effect of several possible processes: restricted movement of polymer chains' segments in the immediate vicinity of solid particles, which leads to a decrease in local conductivity; the influence of the conductive properties of the particles themselves; irregular particle shape. The most remote region in $ \sigma_2 (r) $ effectively takes into account the dependence of $ \sigma_0 $ on the concentration $ c $ of PAAM particles in $ \rm OMPEO-LiClO_4-PAAM $ CPE due to bindings with salt ions of solitary PAAM molecules distributed in the matrix outside the PAAM globules during the process of samples preparation.

Due to the different physical nature of the mechanisms involved, the parameters of these regions must depend differently on the temperature. Because the conductivities of the three regions of the profile $ \sigma_2 (r) $ and the matrix in $ \rm OMPEO-LiClO_4-PAAM $  CPE are formed by processes in regions with different degrees of amorphization, the temperature dependences of these regions' conductivities were modeled using the three-parameter Vogel-Tamman-Fulcher  (VTF) empirical law. The VTF parameters for these regions are found by processing the three isotherms $ \sigma_ {\rm eff} (c, T) $ within a three-shell model with fixed values  of the other model's parameters. It is shown that the obtained values are sufficient for restoration of $ \sigma_ {\rm eff} $ temperature dependences for the rest of the studied CPEs with other  PAAM concentration values.

The model is also used for analysis of electrical percolation in an insulator-conductor type system with inhomogeneous interphase layer having $ \sigma_0 \ll \sigma_2 \leq \sigma_1 $. It is shown that in this case the behavior of the effective electric conductivity $ \sigma _ {\rm eff} $ and dielectric permittivity $ \varepsilon _ {\rm eff} $ is of percolation type.
It is established that within the model the percolation threshold $ c _ {\rm c} $ in the system of penetrable shells is determined only by the geometric properties of the shells.  On the example of model exponential distributions of the shell's conductivity, the percolation behavior of the effective electric conductivity and dielectric permittivity in the vicinity of this threshold is analyzed, and it is shown that the effective critical percolation indices are not universal, but depend on the components' relative conductivities and the concentration interval, on which they are determined. This fact explains a wide range of their known values.
It is shown that the theory can demonstrate the ``double percolation'' effect -- the appearance of the second percolation transition due to direct contacts between highly conductive cores. The dielectric constant in the vicinity of each of the percolation thresholds has a maximum.
It is shown that the model with a homogeneous shell at $ c <c _ {\rm c} $ describes well enough the experimental data for $ \varepsilon _ {\rm eff} $ and $ \sigma _ {\rm eff} $ of a specially prepared system based on KCl with Ag nanoparticles coated by a penetrable oxide shell. For $ c> c _ {\rm c} $ the internal structure of the shells becomes important, therefore the considered inhomogeneous profile is used for restoration of the known data for $ \sigma _ {\rm eff} $. The obtained estimates for the relative shell thickness are close to those predicted by the experimenters; the obtained inhomogeneous structure of the shell's profile may be the result of electron tunneling mechanism, which is confirmed by the shape of its outer part and estimates of the characteristic length of tunneling.

The method of compact groups, used in this work, is used for critical analysis of differential scheme for calculating the effective quasistatic electrical parameters of disperse systems. It is shown that the classical differential scheme is applicable only to systems with small differences in dielectric permittivities of the components and in narrow concentration intervals, and its modifications lead to results that do not satisfy the known strict limits for the values of the effective system's parameters.

At the end of the thesis the main conclusions, made on the basis of the performed dissertation research, and the list of used bibliographic resources are given.

\vskip 15pt
\textbf{Key words:} compact group approach, core-shell model, electric conductivity, dielectric permittivity, disperse system, percolation, composite electrolytes, nanocomposites, differential scheme

\end{otherlanguage*}