\begin{otherlanguage*}{english}
%\part{Abstract}
\begin{center}
    {\normalfont \textbf{
    	ABSTRACT\\
    	Semenov A.K. Electrophysical properties of multiphase disperse systems.} -- Qualification scientific paper, manuscript.}
\end{center}
\vskip 5pt

Thesis for a Candidate of Science in Physics and Mathematics (Philosophy Doctor) degree by specialty 01.04.02 -- Theoretical Physics. -- Odesa I.I.~Mechnikov National University, the MES of Ukraine, Odesa, Ukraine, \the\year.

\vskip 5pt

The thesis is concentrated on constructing a theoretical model for the effective quasistatic electrophysical characteristics of macroscopically homogeneous and isotropic three-dimensional random  systems of spherical particles with hard-core--penetrable-shell morphology, its analysis and applications to description of the quasistatic electric conductivity $ \sigma_{\rm eff}$ and dielectric permittivity $\varepsilon_ {\rm eff} $ of real random heterogeneous systems.
Being the most common, such systems are theoretically  studied most poorly because of the necessity, on one hand, to take into account various interphase and matrix effects that form the microstructure of the system and, on the other hand, to solve an essentially many-particle electrodynamic problem.
It is shown in the present work that their effective description is possible by using the indicated model system of spherical particles with hard-core--penetrable-shell morphology and suggesting that: (1) the shells in general have a radial distribution of the complex dielectric permittivity; (2) the local value of the complex permittivity at the points of  overlappings of the system's components is determined by the distance to the nearest particle.

The model is analyzed with  the method of compact groups of inhomogeneities, generalized to systems with conducting components. According to this method the system is considered as a set of macroscopic regions (compact groups) that are point-like with respect to the testing field, but large enough to have the effective properties of the whole system.
Compact group contributions are formed by the most singular (delta-function) parts of the propagators and are dominant in the quasistatic limit, which allows one to sum up the entire iterative series for the electric field and complex current without the need to calculate its individual terms.
The calculation scheme is closed by the requirement that the known boundary conditions for the normal components of the complex electric field be fulfilled.
The final results are the integral relation, which relates the effective quasistatic complex permittivity of the system with the complex permittivities and volume concentrations of its components, and, under certain conditions, similar relations for $ \sigma_ {\rm eff} $ and $ \varepsilon_ {\rm eff} $.

To test the theoretical results obtained the existing data of numerical simulations with the Random Resistor Network algorithm for the static conductivity of the considered model systems with different core diameters and thicknesses of electrically homogeneous and inhomogeneous shells were used. These data had been obtained for the case when the conductivity of the shells $ \sigma_2 $ is much higher than the conductivity of the matrix $ \sigma_0 $ and that of the cores $ \sigma_1 $, which is typical of composite solid (CSE) and composite polymer  (CPE) electrolytes.
Taking into account the peculiarities of the simulation algorithm  and the related problems of mapping the model's results on the simulation data, it is shown that the theory is able to fully reproduce these data.

It is demonstrated that the model is applicable for describing the concentration dependences of the electric conductivity $ \sigma_ {\rm eff} $ of real  composite solid electrolytes and composite  polymer electrolytes.
The known experimental data by Liang for CSE formed by dispersing  $ \rm Al_2O_3 $ particles into a  $ \rm LiI $ polycrystalline matrix and those by Wieczorek's group for CPEs based on poly(ethylene) oxide ($ \rm PEO $) or oxymethylene-linked  $ \rm PEO $ ($ \rm OMPEO $) with addition of $ \rm NaI $ or $ \rm LiClO_4 $ salts and $ \rm Na_ {3.2} Zr_2P_ {0.8} Si_ { 2.2} O_ {12} $ (NASICON), $ \rm \theta-Al_2O_3 $ particles  or polyacrylamide ($ \rm PAAM $) globules as fillers  were used for the analysis.
The data were processed using model conductivity profiles $ \sigma_2 (r) $ for the penetrable shells, the shape of which was gradually varied from a step of a constant height to a superposition of sigmoids.
The assumptions was made and justified that the profiles $ \sigma_2 (r) $ obtained as a result of such a processing procedure can be used to analyze the role of different physicochemical mechanisms in the formation of the effective conductivity $ \sigma_ {\rm eff} $.

In particular, for the $ \rm LiI-Al_2O_3 $ CSE, $ \sigma_2 (r) $ has two  distinct parts. The outer part incorporates the contribution from matrix processes to the formation of $ \sigma _ {\rm eff} $. These can be uncontrolled doping of the matrix during the  experimental samples preparation, accumulation of dislocations etc.. The inner part indicates the existence of a highly conductive layer around the $ \rm \theta-Al_2O_3 $ particles. It can be caused by accumulation of point defects, as evidenced by the results of comparison of the estimated characteristics of this region with the estimates of other authors.

The results for the CPEs show that there are two or three distinct parts in $ \sigma_2 (r) $, which can be interpreted as follows. The central part of $ \sigma_2 (r) $ signifies the formation around the dispersed particles of amorphous polymer regions with relatively high conductivity resulting from increased mobility of dissolved salt ions in these regions. The region closest to the core incorporates the total effect of several possible processes: hindering of movement of polymer chains' segments in the immediate vicinity of the solid particles, which leads to a decrease in local conductivity; influence of the particles' conductive properties; irregularity of the particles' shape. The outermost part in $ \sigma_2 (r) $ effectively accounts for the dependence of $ \sigma_0 $ on the complexation of salt ions with  solitary PAAM molecules present in the matrix outside the PAAM globules.

Due to different physical nature of the mechanisms involved, the parameters of these regions are expected to depend differently on temperature. Because the conductivities of of these three parts in the profile $ \sigma_2 (r) $ and that of the matrix in $ \rm OMPEO-LiClO_4-PAAM $  CPE are formed by processes in the regions with different degrees of amorphization, their temperature dependences were modeled using the three-parameter Vogel-Tamman-Fulcher  (VTF) empirical law. The VTF parameters for these regions were found by processing three  isotherms $ \sigma_ {\rm eff} (c, T) $ using the three-shell model with fixed values  of the other model's parameters. It is shown that the  values obtained are sufficient to restore the $ \sigma_ {\rm eff} $ versus temperature dependences for the other studied CPEs with differing  PAAM concentration values.

The model is also used to analyze electrical percolation in an insulator-conductor systems with an inhomogeneous penetrable interphase layer of conductivity $\sigma_2$ such that having $ \sigma_0 \ll \sigma_2 \leq \sigma_1 $. It is shown that in this case, the effective electric conductivity $ \sigma _ {\rm eff} $ and dielectric permittivity $ \varepsilon _ {\rm eff} $ demonstrate percolation-type behavior.
The percolation threshold $ c _ {\rm c} $ in the system of penetrable shells is determined only by the geometric properties of the shells.  Using model exponential-type profiles for the shell conductivity, the percolation behavior of the effective electric conductivity and dielectric permittivity in the vicinity of this threshold was analyzed to show that the effective critical percolation exponents are not universal, but depend on the components' relative conductivities and the concentration interval from which they are determined. This fact explains a wide range of their known experimental values.
It is also shown that the theory can demonstrate the ``double percolation'' effect, that is the appearance of the second percolation transition due to direct contacts between highly conductive cores. The dielectric constant in the vicinity of each of the percolation thresholds has a maximum.
It is demonstrated that for $c<c_{\rm c}$, the model with a homogeneous shell describes sufficiently well the experimental data for $ \varepsilon _ {\rm eff} $ and $ \sigma _ {\rm eff} $ of a specially prepared KCl-based composite  system with Ag nanoparticles coated with a penetrable oxide shell. For $ c> c _ {\rm c} $ the internal structure of the shells becomes important, and an inhomogeneous profile was used to restore the experimental data for $ \sigma _ {\rm eff} $. The  estimates for the relative shell thickness are close to those predicted by the experimenters. The estimated inhomogeneous structure of the shell's profile may be a result of the electron tunneling mechanism, which is indicated by the shape of its outer part and estimates of the tunneling characteristic length.

Finally, the method of compact groups is applied to critical analysis of the differential scheme for calculating the effective quasistatic electrical parameters of disperse systems. It is shown that the classical differential scheme is applicable only to systems with small differences in the dielectric permittivities of the components and in narrow concentration intervals, while its modifications lead to results that violate the known strict boundaries for the values of the effective parameters of the system.

At the end of the thesis the main conclusions of the dissertation  research and a list of the bibliographic sources used are given.

\vskip 15pt
\textbf{Key words:} compact group approach, core-shell model, electric conduc\-tivity, dielectric permittivity, disperse system, percolation, composite electro\-lytes, nanocomposites, differential scheme

\end{otherlanguage*}