\begin{otherlanguage*}{english}
%\part{Abstract}
\begin{center}
    {\normalfont \textbf{
    	ABSTRACT\\
    	Semenov A.K. Electrophysical properties of multiphase disperse systems.} -- Qualification scientific paper, manuscript.}
\end{center}
\vskip 15pt

Candidate degree (PhD) thesis in Physics and Mathematics Sciences. Speciality 01.04.02 -- theoretical physics. Odesa I.I. Mechnikov National University, the MES of Ukraine, Odesa, \the\year.

\vskip 15pt

In the thesis we have built a model of quasi-static electric response of disordered three-dimensional particulate systems with a hard-core--penetrable-shell morphology. The shells were in general electrically inhomogeneous. Derivations were based on the compact group of inhomogeneities approach. The known boundary conditions for the normal components of complex fields were used as a closure for the theory.

The theory was tested on existing results of numerical simulations for the systems under consideration. Taking into account the peculiarities of the simulations, it is shown that the theory is able to fully reproduce this data.

Applicability to the real systems was demonstrated for real solid composite and polymer composite electrolytes. Physical interpretation of different parts of the shell�s conductivity profile obtained from fitting the experimental results was discussed.

Analysis of electric percolation behavior in insulator--conductor type systems have shown that within the model the percolation threshold depends only on the relative thickness of the shell, whereas the effective critical indices depend not only on the geometric and electrical parameters of the components, but also on the method of processing the experimental data.

It was also established, that the existing differential scheme of calculating the effective quasi-static electric parameters of dispersed systems is applicable only to systems with small differences in their components� dielectric constants and at narrow concentration intervals.



\vskip 15pt
\textbf{Key words:} compact group approach, core-shell model, electric conductivity, dielectric permittivity, disperse system, percolation, composite electrolytes, nanocomposites, differential scheme

\end{otherlanguage*}