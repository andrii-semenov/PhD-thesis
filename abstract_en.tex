\begin{otherlanguage*}{english}
%\part{Abstract}
\begin{center}
    {\normalfont \textbf{
    	ABSTRACT\\
    	Semenov A.K. Electrophysical properties of multiphase disperse systems.} -- Qualification scientific paper, manuscript.}
\end{center}
\vskip 15pt

Candidate degree (PhD) thesis in Physics and Mathematics Sciences. Speciality 01.04.02 -- theoretical physics. Odesa I.I. Mechnikov National University, the MES of Ukraine, Odesa, \the\year.

\vskip 15pt

We have developed a model for quasi-static electric response of random 3-D  systems of particles with a hard-core--penetrable-shell morphology. The shells were in general electrically inhomogeneous. The derivations were carried out using the method of compact groups of inhomogeneities. The requirement that the known boundary conditions for the normal component of complex fields be valid for the homogenized medium allowed us to close the theory.

The theory was tested using existing simulation data for the systems under consideration. Taking into account the peculiarities of the simulations, it was shown to be capable of reproducing the data fully.

The theory was shown to be applicable for the description of real solid composite and polymer composite electrolytes. A physical interpretation was discussed of different parts of the shell�s conductivity profile obtained by fitting the experimental data.

The theory was also used to analyze electric percolation in insulator/\-conductor  systems. It was shown that within the model, the percolation threshold depends only on the relative thickness of the shell, whereas the effective critical exponents depend not only on the geometric and electrical parameters of the components, but also on the widths of the processed concentration intervals.

It was also shown, that existing differential schemes for calculating the effective quasi-static electric parameters of dispersed systems are applicable only for systems with slightly differing dielectric constants of the components and within narrow concentration ranges.



\vskip 15pt
\textbf{Key words:} compact group approach, core-shell model, electric conductivity, dielectric permittivity, disperse system, percolation, composite electrolytes, nanocomposites, differential scheme

\end{otherlanguage*}