% !TeX spellcheck = uk_UA
%\part{АНОТАЦІЯ}
\begin{center}
    {\normalfont \textbf{
    	АНОТАЦІЯ\\
    	Семенов А.К. Електрофізичні властивості багатофазних дисперсних систем.} -- Кваліфікаційна наукова праця на правах рукопису.}
\end{center}
\vskip 10pt

Дисертація на здобуття наукового ступеня кандидата фізико-матема\-тичних наук за спеціальністю 01.04.02 -- теоретична фізика. -- Одеський національний університет імені І.І. Мечникова, МОН України, Одеса, 2020.

\vskip 10pt

Дисертаційна робота присвячена побудові теоретичної моделі ефективних квазістатичних електрофізичних характеристик невпорядкованих тривимірних макроскопічно однорідних та ізотропних систем сферичних частинок з морфологією тверде ядро-проникна оболонка, її аналізу та застосуванням до опису квазістатичних електричної провідності $\sigma_{\rm eff}$ та діелектричної проникності $\varepsilon_{\rm eff}$ багатофазних гетерогенних систем. 
Ці системи є одними з найбільш поширених, але найменш теоретично досліджених через необхідність, з одного боку, враховувати різноманітні міжфазні та матричні ефекти, що формують мікроструктуру системи та, з другого, розв'язувати суттєво бачаточастинкову електродинамічну задачу.
У роботі показується, що їх ефективний опис можливий при використанні модельної системи сферичних частинок з морфологією тверде ядро - проникна оболонка. Оболонки в загальному випадку мають радіальний розподіл комплексної діелектричної проникності.
Локальне значення  комплексної проникності в точках перекривання компонентів системи визначається відстанню до найближчої частинки. 
Для побудови моделі застосовується метод компактних груп неоднорідностей, який узагальнюється на системи з провідними компонентами. В рамках цього методу система розглядається, як сукупність макроскопічних областей (компактних груп), які є точковими по відношенню до тестуючого поля, але достатньо великими, щоб мати властивості всієї системи. 
Внески компактних груп формують найбільш сингулярні (у вигляді дельта-функцій) частини пропагаторів. Їх внески в квазістатичному наближенні є домінуючими, що дозволяє підсумувати ітераційні ряди для електричного поля і комплексного стуму без необхідності розраховувати їх окремих доданків.
Обчислювальну схему замкнено вимогою виконання в її рамках відомих граничних умов для нормальних компонент комплексного електричного поля.
Остаточними результатами є інтегральне співвідношення яке пов'язує ефективну квазістатичну комплексну проникність системи з комплексними проникностями та об'ємними концентраціями її компонентів та, при виконанні певних умов, схожі співвідношення для $\sigma_{\rm eff}$ та $\varepsilon_{\rm eff}$.

Для тестування теоретичних результатів були використані існуючі дані числових симуляцій, виконаних в рамках алгоритму Random Resistor Network для статичної провідності вказаних модельних систем  з різними діаметрами ядер та товщин електрично однорідних та неоднорідних оболонок. Ці дані були отримані для випадку, коли провідність оболонок $\sigma_2$ є набагато вищою, ніж провідності матриці $\sigma_0$ та ядер $\sigma_1$, який є характерним для твердих композитних (ТКЕ) та полімерних композитних (ПКЕ) електролітів.
Беручи до уваги  особливості  використаного в симуляціях алгоритму та пов'язані з цим проблеми відображення результатів моделі на дані симуляцій, показано, що теорія спроможна повністю відтворити ці дані.

Продемонстровано застосовність моделі для опису концентраційних залежностей електричної провідності $\sigma_{\rm eff}$ реальних твердих композитних та полімерних композитних електролітів.
Для аналізу використовувалися відомі експериментальні дані Ліанга для ТКЕ, утвореного диспергуванням частинок $\rm Al_2O_3$
в полікристалічну матрицю $\rm LiI$, та групи Вєчорека для ПКЕ на основі поліетилен-оксиду ($\rm PEO$)
та $\rm PEO$ з приєднаним оксіметиленом ($\rm OMPEO$) з додаванням солей $\rm NaI$ або $\rm LiClO_4$, де в якості наповнювачів виступали частинки $\rm Na_{3.2}Zr_2P_{0.8}Si_{2.2}O_{12}$ (NASICON), $\rm \theta-Al_2O_3$ або глобули поліакриламіду ($\rm PAAM$). 
Обробка даних виконувалась за допомогою модельних профілів  провідності $\sigma_2(r)$ проникних оболонок, форма яких поступово ускладнювалася від сходинки зі сталою висотою до суперпозиції сигмоїд.
Зроблено і аргументовано припущення, що отримані за результатами такої обробки профілі $\sigma_2(r)$ можуть бути використані для аналізу ролі різних фізико-хімічних механізмів у формуванні ефективної провідності $\sigma_{\rm eff}$. 

Зокрема, для ТКЕ $\rm LiI-Al_2O_3$ $\sigma_2(r)$ має дві чітко виділені ділянки.   Зовнішня ділянка відображає внесок матричних процесів у формування $\sigma_{\rm eff}$. Ними можуть бути неконтрольоване легування матриці при підготовці експериментальних зразків, накопичення дислокацій тощо. Внутрішня ділянка вказує на існування високопровідного шару навколо частинок $\rm \theta-Al_2O_3$. Він може спричинятися накопиченням точкових дефектів, про що свідчать результати порівняння
отриманих характеристик цієї ділянки з оцінками інших авторів.

Результати для ПКЕ показують наявність двох-трьох чітко виражених
ділянок $\sigma_2(r)$, які допускають наступну інтерпретацію. Центральна ділянка
$\sigma_2(r)$ відображає формування навколо частинок в ПКЕ аморфізованих областей з відносно високою провідністю, яка є результатом підвищеної рухливості іонів розчиненої солі в цих областях. Найближча до ядра ділянка описує сумарний ефект кількох можливих процесів: утруднення руху сегментів полімерних ланцюгів в безпосередньому околі твердих частинок, що веде до зниження локальної провідності; вплив провідних властивостей самих частинок;
нерегулярність форми частинок. Найвіддаленіша ділянка $\sigma_2(r)$ ефективно
відображає залежність $\sigma_0$ від концентрації $c$ частинок PAAM в ПКЕ $\rm OMPEO-LiClO_4-PAAM$ внаслідок зв'язування іонами солі поодиноких молекул PAAM, розподілених в матриці поза межами глобул PAAM в процесі створення зразків.

У силу різної фізичної природи задіяних механізмів параметри цих ділянок повинні по-різному залежати від температури. Оскільки
провідності трьох ділянок профілю $\sigma_2(r)$ та матриці в ПКЕ $\rm OMPEO-LiClO_4-PAAM$ формуються процесами в областях з різним ступенем аморфності, то температурні залежності провідностей цих областей моделювались за допомогою трипараметричного емпіричного закону Фогеля-Таммана-Фульхера (VTF). Параметри VTF для цих областей знаходяться шляхом обробки трьох ізотерм $\sigma_{\rm eff} (c,T)$ в рамках тришарової моделі при фіксованих значеннях інших параметрів моделі. Показано, що отриманих значень достатньо для
відновлення температурних залежностей $\sigma_{\rm eff}$ для решти досліджених ПКЕ з іншими значеннями концентрації PAAM.

Модель також застосовано для аналізу електричної перколяції в системі типу ізолятор-провідник з неоднорідним проникним міжфазним шаром при $\sigma_0 \ll \sigma_2 \leq \sigma_1$. Показано, що у цьому випадку поведінка
ефективних електричної провідності $\sigma_{\rm eff}$ та діелектричної проникності $\varepsilon_{\rm eff}$ має перколяційний характер.
Поріг перколяції $c_{\rm c}$ в системі проникних оболонок визначається лише геометричними властивостями оболонок. На прикладі модельних експоненціальних розподілів провідності оболонок проаналізовано перколяційну поведінку ефективних електричної провідності та діелектричної проникності в околі цього порогу та показано, що  ефективні  критичні індекси перколяції не є універсальними, а залежать від відносних провідностей компонентів та концентраційного інтервалу, на якому вони визначаються. Цей факт дозволяє пояснити широкий спектр їх відомих експериментальних значень. 
Також показано, що теорія може демонструвати ефект ``подвійної перколяції'' -- появу другого перколяційного переходу за рахунок безпосередніх контактів між високопровідними ядрами. Діелектрична проникність в околі кожного з порогів перколяції має максимум. 
Продемонстровано, що модель з однорідною оболонкою при $c<c_{\rm c}$ достатньо добре описує експериментальні дані для $\varepsilon_{\rm eff}$ та $\sigma_{\rm eff}$ спеціально підготовленої композитної системи на основі KCl
з наночастинками Ag, покритими проникним оксидним шаром. При $c>c_{\rm c}$ важливою стає внутрішня структура оболонок, тому для відновлення наявних даних для $\sigma_{\rm eff}$ використано розглянутий неоднорідний профіль. Отримані оцінки для відносної товщини оболонки близькі до прогнозованих експериментаторами; отримана неоднорідна структура профілю оболонки може бути результатом механізму тунелювання електронів, на що вказують форма його зовнішньої частини та оцінки характерної довжини тунелювання. 

Використаний в роботі метод компактних груп застосовано для критичного аналізу диференціальної схеми обчислення ефективних квазістатичних електричних параметрів дисперсних систем.  Показано, що класична диференціальна схема є застосовною лише для систем з малими різницями діелектричних проникностей компонентів та у вузьких концентраційних інтервалах, а її модифікації ведуть до результатів, що не задовольняють відомі строгі межі   для значень ефективних параметрів системи.

Наприкінці роботи наведено основні висновки, зроблені на базі виконаного дисертаційного дослідження, та список використаних бібліографічних джерел.

\vskip 15pt
\textbf{Ключові слова:} метод компактних груп, модель ядро-оболонка, електрична провідність, діелектрична проникність, дисперсна система, перколяція, композитні електроліти, нанокомпозити, диференціальний метод
