% \iffalse
% !Mode:: "DTX:UNIX:UTF-8"
%
% This is the source file for the GOST package
% to be used with BibTeX8.
% It contains set of bibliographic style that
% attempt to format the bibliography according to
% GOST R 7.0.5-2008 with some conventional modifications.
%
% Copyright 2012-2017 Igor A. Kotelnikov.
% Version 1.2i
%
% Copyright 1996-2005 Maksym Polyakov.
% Version 2005.08.12
%
% For a bibliographic entry in Russian you need
% to add the field
%     language = {russian}.
% Ukrainian, German, French, Italian languages are
% also partially supported. Default language is English.
%
% Book entry should have the pagetotal field
% showing total number of pages. Optional field
% url will format URL (using hyperref, if loaded).
%
% See examples in gost*.pdf.
%
% Requires 8-bit bibtex with switch -B and csf file supporting cyrillic.
% bibtex8 -B -c <encoding>.csf <your_document>
%
%
% Download from
%  CTAN:macros/latex/contrib/gost/gost.dtx
%
% To unpack run
%   latex gost.ins
%
% To produce the documentation run
%    pdflatex gost.dtx
%    makeindex -r -s gind.ist gost
%    makeindex -r -s gglo.ist -o gost.gls gost.glo
%    pdflatex gost.dtx
%    pdflatex gost.dtx
%
% TeX Directory Structure
%
% Move all *.bst files       to TDS:bibtex/bst/gost
% Move all *.csf files       to TDS:bibtex/csf/gost
% Move all *.pdf files       to TDS:doc/bibtex/gost
% Move gost.dtx and gost.ins to TDS:source/bibtex/gost/
%
% $Id: gost.dtx, v.2017/01/12 kia999 $
%
%<*driver>
\ProvidesFile{gost.dtx}%
  [2017/01/12 v.1.2i Bibliography styles for GOST R 7.0.5-2008.]

\documentclass{ltxdoc}

\usepackage{geometry}
\usepackage{makeidx}
\ifpdf
  \usepackage[pdftex,hyperindex,unicode]{hyperref}
  \usepackage{cmap}
\else
  %\usepackage[hypertex]{hyperref}
  \usepackage{hyperref}
\fi
\usepackage{color}
\definecolor{darkblue}{rgb}{0,0,.6}
\hypersetup{
  pdftitle           = {The GOST package},
  pdfauthor          = {Igor A. Kotelnikov},
  pdfsubject         = {BibTeX, GOST},
  pdfstartview       = {FitH},
%  pdfborder          = {0 0 0},
%  bookmarksopen      = true,
  bookmarksnumbered  = true,
%  bookmarksopenlevel = 2,
  colorlinks = true,     linkcolor  = darkblue,
  citecolor  = darkblue, filecolor  = darkblue,
  menucolor  = darkblue, urlcolor   = darkblue
}


\usepackage[utf8]{inputenc}
\usepackage[T2A]{fontenc}
\usepackage[russian,english]{babel}

\MakeShortVerb{\|}

\def\file#1{\texttt{#1}}
\def\pkg#1{\textsf{#1}}
\def\cmd#1{\texttt{#1}}

\newcommand\DescribeEntry[1]{%
  \DescribeMacro{#1}\index{Entry!\texttt{#1}}%
}
\newcommand\DescribeFunction[1]{%
  \DescribeMacro{#1}\index{\texttt{#1}}%
}
\newcommand\DescribeField[1]{%
  \DescribeMacro{#1}\index{Field!\texttt{#1}}%
}


\GetFileInfo{gost.dtx}
\EnableCrossrefs
%\DisableCrossrefs   % Say \DisableCrossrefs if index is ready
\RecordChanges       % Gather update information
\CodelineIndex       % Index code by line number
%\PageIndex           % Index code by page number

\title{The GOST package}

\author{
    Igor A. Kotelnikov\thanks{\texttt{kia999@mail.ru}}
}

\date{\filedate, \fileversion}

\makeindex

\begin{document}
  \maketitle
  \DocInput{gost.dtx}
\end{document}
%</driver>
% \fi
%
% \CheckSum{592}
%
% \DoNotIndex{\cyra,\cyrb,\cyrv,\cyrg,\cyrd,\cyre}
% \DoNotIndex{\CYRP,\cyrii,\cyrr,\cyre,\cyro,\CYRU,\cyrk,\cyrl,\CYRS,\cyrs|}
% \DoNotIndex{\cyrt,\cyri,\cyrz,\CYRT,\cyrm}
% \DoNotIndex{\cyr,\cyrc,\CYRD,\cyrery,\cyrh,\CYRK,\CYRM,\CYRN,\cyrn}
% \DoNotIndex{\cyrp,\cyrs,\cyrshch,\cyru,\CYRV,\cyrya}
% \DoNotIndex{\CYRA,\CYRB,\cyrch,\CYRE,\CYREREV,\CYRF,\cyrf,\CYRI,\cyrishrt}
% \DoNotIndex{\CYRO,\CYRR,\cyrsftsn,\CYRYA,\cyryu,\cyrzh}
% \DoNotIndex{\z@skip,\z@}
% \DoNotIndex{\|,\~,\ }
% \DoNotIndex{\\&,\',\(,\),\`,\\,\/,\.}
% \DoNotIndex{\begin,\DeclareUTFcharacter}
% \DoNotIndex{\csname,\def}
% \DoNotIndex{\else,\endcsname,\end}
% \DoNotIndex{\expandafter,\fi}
% \DoNotIndex{\hbox,\hss}
% \DoNotIndex{\iffalse,\ifx,\ignorespaces,\iflanguage}
% \DoNotIndex{\language,\lastskip,\ldots,\let,\natexlab,\newblock,\nobreak}
% \DoNotIndex{\providecommand,\ProvideTextCommandDefault,\relax,\renewcommand}
% \DoNotIndex{\selectlanguage,\selectlanguageifdefined,\small,\textemdash}
% \DoNotIndex{\textnumero,\unskip,\url,\UTFencname}
% \DoNotIndex{\,,\catcode,\cite,\citeauthor,\citet,\citeyear,\href,\hskip,\ifdim}
%
% \begin{abstract}
%
% GOST is a bundle of BibTeX styles designed to meet State Standards (GOST)
% on information, librarianship and publishing issued by The Russian Federation
% and Interstate Committee of former USSR States.
%
% It comprises 16 BibTeX styles to format bibliography in English, Russian
% and Ukrainian according to GOST 7.0.5-2008 and GOST 7.1-2003. Both 8-bit
% and Unicode (UTF-8) versions of each BibTeX style, in each case offering
% a choice of sorted and unsorted. Further, 2 more styles, |gost780| and
% |gost780s| styles (not conforming to effective standards) are retained
% for backwards compatibility.
%
% \end{abstract}
%
% \section{Introduction}
%   The package was initially developed by Maksym Polyakov. It was later
%   updated by Igor Kotelnikov to the present status and some code was borrowed
%   from \pkg{disser} package developed by Stanislav Kruchinin and unpublished
%   work by Artem Petrenkov.
%
%   Nowdays, GOST is a bundle of BibTeX styles designed to meet State Standards
%   (GOST) on information, librarianship and publishing issued by
%   Russian Federation and interstate committee of former USSR States.
%
%   The System of Standards includes:
%   \begin{description}
%   \item[GOST 7.0.5-2008]
%                Bibliographic reference.
%                General requirements and rules of making.
%   \item[GOST 7.1  -2003]
%                Bibliographic record.
%                Bibliographic description.
%                General requirements and rules.
%   \item[GOST 7.80 -2000]
%                Bibliographic record.
%                Heading. General requirements and rules.
%   \item[GOST 7.11 -2004]
%                Bibliographic description and references.
%                Rules for the abbreviation of words and word combinations
%                in foreign European languages.
%   \item[GOST 7.83 -2001]
%                Electronic editions. Basic types and imprint.
%   \end{description}
%   ect.
%
%   Currently, GOST contains 16 BibTeX styles to format bibliography in English,
%   Russian and Ukrainian according to GOST 7.0.5-2008 and GOST 7.1-2003. Both
%   8-bit and Unicode (UTF-8) versions of each BibTeX style, in each case
%   offering a choice of sorted and unsorted. Further, 2 styles, |gost780| and
%   |gost780s| styles (which do not conform to effective standards) are retained
%   for backwards compatibility.
%
%   All styles in the GOST bundle are derived from single master file
%   \file{gost.dtx} by applying different set of options as shown in the table
%   below.
%
%   \begin{verbatim}
%------------------------------------------------------------------------
% Style         | utf8 | strict | modern | eprint | long | sort | natbib
%------------------------------------------------------------------------
% gost780       |      |        |        |        |      |      |
% gost780s      |      |        |        |        |      |   +  |
%------------------------------------------------------------------------
% gost2003      |      |   +    |        |    +   |      |      |
% gost2003s     |      |   +    |        |    +   |      |   +  |
% gost2008      |      |        |   +    |    +   |      |      |
% gost2008n     |      |        |   +    |    +   |      |      |   +
% gost2008l     |      |        |   +    |    +   |   +  |      |
% gost2008s     |      |        |   +    |    +   |      |   +  |
% gost2008ns    |      |        |   +    |    +   |      |   +  |   +
% gost2008ls    |      |        |   +    |    +   |   +  |   +  |
%------------------------------------------------------------------------
% ugost2003     |  +   |   +    |        |    +   |      |      |
% ugost2003s    |  +   |   +    |        |    +   |      |   +  |
% ugost2008     |  +   |        |   +    |    +   |      |      |
% ugost2008n    |  +   |        |   +    |    +   |      |      |   +
% ugost2008l    |  +   |        |   +    |    +   |   +  |      |
% ugost2008s    |  +   |        |   +    |    +   |      |   +  |
% ugost2008ns   |  +   |        |   +    |    +   |      |   +  |   +
% ugost2008ls   |  +   |        |   +    |    +   |   +  |   +  |
%------------------------------------------------------------------------
% Style         | utf8 | strict | modern | eprint | long | sort | natbib
%------------------------------------------------------------------------
%   \end{verbatim}
%
% |Gost2008| style is recommended for most applications. It corresponds to the
% currently effective Standard 7.0.5-2008. Librarians should use the style
% |gost2003| instead of |gost2008| to compile a library catalog to meet the
% Standard 7.1-2003. Use of other styles is best explained through
% the meaning of options used to compile those styles from the master source.
%
% Two styles, |gost780| and |gost780s|, compiled without |modern| and |strict|
% options, are retained for backward compatibility. They do not conform
% the Standards 7.0.5-2008 and 7.1-2003 cited above.
%
% The |strict| option provides conformance to the Standard 7.1-2003. The
% styles compiled with that option bear the name |gost2003| with possible
% suffixes |s|, |l|, |n| as explained below. These styles are intended
% primarily for the librarians who compose a library catalog.
%
% The |modern| option meets the Standard 7.0.5-2008 which can be thought
% off as a relaxed version of the Standards 7.1-2003. The styles compiled with
% that option bear the name |gost2008| with possible suffixes |s|, |l|, |n|.
% The |strict| option  has precedence over |modern| so that a style compiled
% with both options will resemble mainly the |gost2003| style rather than
% |gost2008|.
%
% If the number of authors exceeds 4, modern styles cut the list of authors
% to at most 4 persons as prescibed by the Standards. Option
% |long| overrides this rule to provide backward compatibility with the
% package disser by Stanislab Kruchinin. Two styles, |gost2008l| and
% |gost2008ls|, compiled with the option |long| mimic behavior of the styles
% |gost705| and |gost705s| from the disser package. Major
% effect of the |long| option is that the list of authors always
% precedes book or article title no matter how long is it.
% Modern styles compiled without |long| place long list of authors behind
% the title. The names of styles compiled with the option |long| has the
% suffix |l|. Recall that those styles do not conform effective Standards and
% their use is discouraged.
%
% The |eprint| option enables formatting electronic publications. In particular,
% it enables |eprint|, |eprinttype|, |eprintclass|, and |doi| fields for a
% bibliographic entry. The styles generated without the |eprint| option,
% ignore the these fields. Starting from the
% version 1.2 of the GOST package, all modern styles are compiled with this
% option included, and the suffix |e| which designated
% this option in earlier  versions is not appended to the name of style
% any more.
%
% The |natbib| option provides compatibility with the |natbib| package. The names
% of styles compiled with the option |natbib| bear the suffix |n|. Currently
% 4 styles with that option are available for beta testing.
%
% The |sort| option enables sorting bibliographic references by author names
% and references titles. The names of styles compiled with the option |sort|
% bear the suffix |s|.
%
% Finally, the |utf8| option produces bibliographic styles in unicode rather
% that in 8-bit encoding. Names of those styles bear the prefix |u|.
%
%
% Beyond bibliographic style, GOST bundle contains CS files (codepage and
% sorting order).
%   \begin{verbatim}
%  ----------------------------------------------------------------
%  Encoding        |   CSF                 Sorting order
%  ----------------------------------------------------------------
%  cp866           |   ruscii.csf          Cyrillic first, Latin
%  cp1251          |   cp1251.csf          Cyrillic first, Latin
%  koi8-u          |   koi8u.csf           Cyrillic first, Latin
%  utf8            |   utf8cyrillic.csf    Cyrillic first, Latin
%  ----------------------------------------------------------------
%   \end{verbatim}
%
%  In addition, BibTeX8 distribution comes with few more CSFs.
%   \begin{verbatim}
%  ----------------------------------------------------------------
%  Encoding        |   CSF                 Sorting order
%  ----------------------------------------------------------------
%  cp866           |   cp866rus.csf        Latin first, Cyrillic
%  ----------------------------------------------------------------
%   \end{verbatim}
%
%   \subsection{How to use}
%
%   \begin{enumerate}
%   \item
%   Select bibliography style by adding appropriate |\bibliographystyle|
%   declaration to your source file \file{<filename>.tex}, e.g.
%   \begin{verbatim}
%   \bibliographystyle{gost2008}
%   \bibliography{database}
%   \end{verbatim}
%
%   \item
%   Add the field |language="ukrainian"| or |language="russian"| to the
%   bibliographic entries in Ukrainian or Russian languages in your
%   database; English is the default language. German, Italian and
%   French are partially supported.
%
%   \item
%   To compile list of references from your database use \file{bibtex8.exe}
%   rather than \file{bibtex.exe}. Depending on the codepage of your
%   bibliographic database, indicate one of the CS files
%   listed above as option to \file{bibtex8.exe}.  Run LaTeX, then run
%   \file{bibTeX8} and LaTeX again:
%   \begin{verbatim}
%   latex <filename>.tex
%   bibtex8 -B -c <csf_file>.csf <filename>.aux
%   latex <filename>.tex
%   \end{verbatim}

%   \item
%   For details on preparing bibliographic database see examples in
%   \file{gost*.pdf}.

%   \item
%   \file{ugost*} styles are primarily intended for use
%   with unicode compilers (\file{xelatex} and \file{lualatex}). They
%   should be preferred as well when using 8bit compilers
%   (\file{latex} and \file{pdflatex}) if source file is in utf8 encoding.
%
%   \item
%   Neither \file{bibtex.exe} nor \file{bibtex8.exe} provides
%   correct sorting order of unicode text. It means that using
%   \file{ugost2008s} or \file{ugost2008ns} may produce unexpected result
%   for documents in \texttt{utf8} encoding.
%
%   \item
%   \file{Bibtex8} fails to change case of a string if it contains Cyrillic
%   letter in unicode. Therefore \file{ugost2008*} styles do not change case of
%   titles and other parts of bibliographic record while 8-bit styles
%   do the case change where appropriate.
%
%   \item
%   Either \file{bibtex8} or \file{Bibtex8} fail to cut Cyrillic names to
%   initials. Therefore \file{ugost2008*} styles do not modify name of
%   authors.
%
%   \item
%   Package |natbib| is required when choosing styles with
%   suffix |n| in their names.
%
%   \end{enumerate}
%
%
%   \subsection{Customization}
%
%   Every GOST style defines few commands to format some parts of a
%   reference. You can redefine these commands prior to
%   the |\bibliography{<bibtex_style>}| command. Initial
%   definitions are listed below.
%   \begin{verbatim}
%   \providecommand*{\url}[1]{{\small #1}}
%   \providecommand*{\BibUrl}[1]{\url{#1}}
%   \providecommand{\BibAnnote}[1]{}
%   \providecommand*{\BibEmph}[1]{#1}
%   \end{verbatim}
%
%   By default, gost styles separate logical parts of a bibliography
%   record by a period and cyrdash (|. "---|). It is legitimate to drop
%   that dash by overriding the command |\BibDash| as follows
%
%   \begin{verbatim}
%   \providecommand*{\BibDash}{}
%   \end{verbatim}
%
%   By default, |\BibDash| is equivalent to the shorthand |"---|
%   defined by the babel package with the option |russian|.
%   It prints a so called Cyrillic dash (|\cyrdash|),
%   which is 20\% shorter then ordinary LaTeX dash (|---|), and puts
%   unbreakable space before |\cyrdash| so that dash never appears
%   in the beginning of a line.
%
%   \subsection{Where to get}
%
%   \begin{enumerate}
%   \item
%   \href{http://mirror.ctan.org/biblio/bibtex/contrib/gost/}
%       {CTAN:biblio/bibtex/contrib/gost}.
%   \item
%   \href{http://ctan.org/pkg/gost}
%       {CTAN:pkg/gost}.
%   \end{enumerate}
%
%
%   \subsection{Version history}
%
%   \subsubsection*{Version 1.2i (2017.01.12)}
%
%   \begin{enumerate}
%   \item
%   Documentation and examples update (thanks to Leonid Sinev).
%
%   \item
%   Restored |@MastersThesis| instead of |@MasterThesis| (thanks to Leonid Sinev).
%
%   \item
%   |media="eresource"| is introduced in addition to |media="online"| and
%   |media="text"|; if present, the |media| field is not ignored any more in
%   modern bst-styles compiled without the |strict| option.
%
%   \item
%   |location| field is introduced as an alias of |address| field.
%
%   \item
%   |@DSCISTHESIS| entry renamed to |@DOCTHESIS|.
%
%   \item
%   |school| field in |@THESIS| and similar entries is replaced by
%   |institution| to comply with |biblatex-gost| style.
%
%   \end{enumerate}
%
%
%   \subsubsection*{Version 1.2h (2016.08.21)}
%
%   \begin{enumerate}
%   \item
%   Minor changes in documentation.
%
%   \end{enumerate}
%
%
%   \subsubsection*{Version 1.2g (2016.07.25)}
%
%   \begin{enumerate}
%   \item
%   Minor changes in documentation.
%
%   \end{enumerate}
%
%
%   \subsubsection*{Version 1.2f (2016.07.12)}
%
%   \begin{enumerate}
%   \item
%   Support for patent entry added (thanks to Stanislav Kruchinin).
%
%   \item
%   |medium| field renamed to |media| field for compatibility with biblatex.

%   \end{enumerate}
%
%   \subsubsection*{Version 1.2e (2016.07.07)}
%
%   \begin{enumerate}
%   \item
%   Hard coded "URL" string replaced with a language sensitive string
%   (thanks to Roman Budnyi).
%
%   \end{enumerate}
%
%   \subsubsection*{Version 1.2d (2015.02.18)}
%
%   \begin{enumerate}
%   \item
%   jan, feb, etc. macros fixed.
%
%   \item
%   New macro format.month.
%   \end{enumerate}
%
%   \subsubsection*{Version 1.2c (2015.01.10)}
%
%   \begin{enumerate}
%   \item
%   |langid| field is added. It has same meaning as |language| which is now
%   obsolete but is still supported for backward compatibility; |langid| has
%   priority over |language|.
%
%   \item
%   |eid| field is added. It has priority over |pages|.
%
%   \item
%   The ligature |"---| has been substituted with |\BibDash| for \texttt{.bst}
%   styles compiled without |modern| options (\texttt{gost2003.bst} and
%   \texttt{gost2003s.bst}). For modern styles this was done in earlier versions.
%
%   \item
%   Spacing around |\BibDash| has been improved.
%
%   \item
%   |\BibDash| now typesets short em-dash (|\cyrdash|) only for |russian| and
%   |ukrainian| languages.  In earlier versions, it produces short em-dash for
%   all languages.
%
%   (This feature was removed since it did not work with all engines.)
%
%   \end{enumerate}
%
%   \subsubsection*{Version 1.2a (2012.08.31)}
%
%   \begin{enumerate}
%
%   \item
%   |\cyrdash| is now defined via |\ProvideTextCommand| rather than
%   |\providecommand|.
%
%   \end{enumerate}
%
%   \subsubsection*{Version 1.2 (2012.02.22)}
%
%   \begin{enumerate}
%   \item
%   Code refactoring. All styles are now generated from single source file.
%
%   \item
%   Support for GOST 7.1-2003. The field |medium| is added to reflect type
%   of material. For most entry types |medium| defaults to |text|.
%
%   \item
%   Support for |natbib| package.
%
%   \item
%   All modern styles are now compiled with the |eprint| option.
%
%   \end{enumerate}
%
%
%   \subsubsection*{Version 1.1 (2012.01.21)}
%
%   \begin{enumerate}
%   \item
%   Support for GOST 7.0.5-2008 and GOST 7.1-2003 is provided.
%
%   \item
%   |@Online| entry is added to format a reference to electronic
%      resource on Internet.
%
%   \item
%   |@MastersThesis| entry is added to format a reference to master's thesis.
%
%   \item
%   |@DSciThesis| entry is added to format a reference to doctor of
%   sciences thesis.
%
%   \item
%   |Urldate|, |eprint|, |eprintclass|, |eprinttype| fields are added.
%
%   \end{enumerate}
%
%   \subsubsection{Older versions}
%
%   \begin{enumerate}
%   \item[2012.02.22] Support for |natbib| package.
%   \item[2012.02.02] Adaptation to GOST 7.0.5, electronic publishing.
%   \item[2005.08.12] First version uploaded to CTAN.
%   \item[2003.06.06] First public version.
%   \end{enumerate}
%
% \StopEventually{
%   \IndexPrologue{\section{Index}
%       Numbers written in dark blue refer to the page where the corresponding
%       entry is described; numbers in black roman refer to the code lines where
%       the entry is used.
%   }
%   \GlossaryPrologue{\section{Change History}}
%   \PrintChanges
%   \PrintIndex
% }
%
% \section{Implementation}
%
%   We need Russian fonts to produce documentation of the code below.
%   Therefore we switch current language to Russian by issuing the command
%   |\selectlanguage{russian}|.
%
% \selectlanguage{russian}
%
%    \begin{macrocode}
%<*bst>
%%
%% This bibstyle attempts to format bibliography according to
%<!modern>%% GOST 7.1-2003 for bibliographic records.
%<modern>%% GOST 7.0.5-2008 for bibliographic reference.
%<*natbib>%%
%%-------------------------------------------------------------------
%% This is an author-year citation style bibliography.
%% It requires a special package file to function properly
%% such as natbib.sty by Patrick W. Daly.
%% The form of the \bibitem entries is
%%   \bibitem[Jones et al.(1990)]{key}...
%%   \bibitem[Jones et al.(1990)Jones, Baker, and Smith]{key}...
%% where the label part (in brackets) consists of the author names,
%% as they should appear in the citation, with the year in parentheses following.
%% There must be no space before the opening parenthesis!
%% A full list of authors may also follow the year.
%% In natbib.sty, it is possible to define the type of enclosures that is
%% really wanted (brackets or parentheses), but in either case, there must
%% be parentheses in the label.
%% The \cite command functions as follows:
%%   \citet{key}              => Jones et al. (1990)
%%   \citet*{key}             => Jones, Baker, and Smith (1990)
%%   \cite{key}              => (Jones et al., 1990)
%%   \cite*{key}             => (Jones, Baker, and Smith, 1990)
%%   \cite[chap. 2]{key}     => (Jones et al., 1990, chap. 2)
%%   \cite[e.g.][]{key}      => (e.g. Jones et al., 1990)
%%   \cite[e.g.][p. 32]{key} => (e.g. Jones et al., p. 32)
%%   \citeauthor{key}         => Jones et al.
%%   \citeauthor*{key}        => Jones, Baker, and Smith
%%   \citeyear{key}           => 1990
%%---------------------------------------------------------------------
%</natbib>

%    \end{macrocode}
% \subsection{Fields}
%
% Enlist all entry types allowed in a bibliographic database.
% Most entries are common for many standard bst styles.
% \changes{v1.2f}{12 Jul 2016}{entry \texttt{patent} added}
% \changes{v1.2c}{18 Jan 2015}{entry field \texttt{langid} added}
% \changes{v1.2c}{18 Jan 2015}{entry field \texttt{eid} added}
% \changes{v1.2}{01 Feb 2012}{entry field \texttt{medium} added}
% \changes{v0.8}{06 Jun 2003}{entry field \texttt{annote} added}
%    \begin{macrocode}
ENTRY
  { address
    annote
    author
    booktitle
    bookauthor
    chapter
    edition
    editor
    compiler
    howpublished
    institution
    journal
    key
    %major          % new in v.1.2i, alias for speciality, not implemented
    majorcode       % new in v.1.2i, alias for specialitycode
    month
    note
    number
    organization
    pages
    eid             % new in v1.2c
    publisher
    school          % alias for institution
    series
    %speciality     % new in v.1.2i, eqv. to major in biblatex-gost, not implemented yet...
    specialitycode  % new in v.1.2i, alias of number, eqv. to majorcode in biblatex-gost
    title
    %medium         % new in v1.2; renamed to media.
    media           % new in v1.2f
    type
    volume
    year
    language
    langid          % new in v1.2c
    booklanguage
%    \end{macrocode}
%   Entries borrowed from biblatex.
% \changes{v1.2b}{19 Jan 2014}{entry field \texttt{numpages} renamed to \texttt{pagetotal}}
% \changes{v1.2}{26 Jan 2012}{entry fields \texttt{eprint}, \texttt{eprintclass}, \texttt{eprinttype} added}
%    \begin{macrocode}
    %date            % not implemented yet...
    pagetotal
    url
    urldate
    isbn
    doi
    eprinttype      % = archivePrefix
    eprintclass     % = primaryClass
    eprint
    % new in v1.2f:
    % appear in biblatex:
    %addendum       % not implemented yet...
    holder          % see patent
    location        % new in v.1.2i, alias of address
    %subtitle       % not implemented yet...
    titleaddon      % new in v.1.2i, see @thesis
    %version        % not implemented yet...
    % Appear in biblatex-gost for @patent entry:
    authorcountry   % country of the patent authors
    credits         % statement of responsibility, other than provided in Biblatex
    ipc             % Code of the International Patent Classification
    %media          % General material designation NOTE: medium in the above
    requestnumber   % Registration number of the application to the patent document
    publicationdate % Date of publication
    publication     %  and information on the official gazette, which published patent
    prioritydate    % Information about the convention priority: the date of filing of the application,
    prioritynumber  %   number and
    prioritycountry %   country name of convention priority.
    requestdate     % ??
  }
  {}
%<!natbib>  { label }
%<natbib>  { label extra.label sort.label short.list }

%    \end{macrocode}
% Declare internal variables and constants used in to format references.
%    \begin{macrocode}
INTEGERS { output.state before.all mid.sentence after.sentence after.block
after.dblslash after.slash after.colon after.semicolon }

%    \end{macrocode}
% \DescribeFunction{init.state.consts}%
%    \begin{macrocode}
FUNCTION {init.state.consts}
{ #0 'before.all :=
  #1 'mid.sentence :=
  #2 'after.sentence :=
  #3 'after.block :=
  #4 'after.dblslash :=
  #5 'after.slash :=
  #6 'after.colon :=
  #7 'after.semicolon :=
}

STRINGS { s t }

STRINGS { curlanguage }

%    \end{macrocode}
%
% \subsection{Formatting functions}
%
% \DescribeFunction{change.language}
% Declare function to switch language.
%    \begin{macrocode}
FUNCTION {change.language}
{ booklanguage empty$
    { "" }
    { booklanguage  'curlanguage :=
      "\selectlanguageifdefined{"
      curlanguage *
      "}" *
    }
  if$
}

%    \end{macrocode}
%
% Declare functions to output various parts of bibliographic record.
% \DescribeFunction{output.nonnull}
% Function |output.nonnull| writes to output buffer the last literal in the stack
% assuming that it is not empty and adds an appropriate punctuation symbol.
% \changes{v1.2c}{17 Jan 2015}{fix Gost2003: \texttt{"}\texttt{-}\texttt{-}\texttt{-} replaced by macro \cs{BibDash}}
%    \begin{macrocode}
FUNCTION {output.nonnull}
{
  swap$
  output.state mid.sentence =
    { ", " * write$ }
    { output.state after.block =
        { add.period$ write$
          " \BibDash " write$
          newline$
          "\newblock " write$
        }
        { output.state before.all =
            'write$
            { output.state after.dblslash =
                { "~//" * change.language * " " * write$ }
                { output.state after.slash =
                    { "~/ " * write$ }
                    { output.state after.colon =
%<!(strict|modern)>                        { ": " * write$ }
%<strict|modern>                        { "~: " * write$ }
                        { output.state after.semicolon =
%<!(strict|modern)>                            { "; " * write$ }
%<strict|modern>                            { "~; " * write$ }
	                        { add.period$ " " * write$ }
    	                  if$
                        }
                      if$
                    }
                  if$
                }
              if$
            }
          if$
        }
      if$
      mid.sentence 'output.state :=
    }
  if$
}

%    \end{macrocode}
%
% \DescribeFunction{output}
% Function |output| calls |output.nonnul| is the last literal string in the stack
% if it is not empty; otherwise it discards the literal.
% Function |output.check| do the same but also warns if the indicated field is
% empty.
% \DescribeFunction{output.check}
%    \begin{macrocode}
FUNCTION {output}
{ duplicate$ empty$
    'pop$
    'output.nonnull
  if$
}

FUNCTION {output.check}
{ 't :=
  duplicate$ empty$
    { pop$ "empty " t * " in " * cite$ * warning$ }
    'output.nonnull
  if$
}

%    \end{macrocode}
%
% \DescribeFunction{fin.entry}
%   |fin.entry| finalizes current entry. It writes dot, if no dot is found in
%   stack, and starts new line.
%    \begin{macrocode}
FUNCTION {fin.entry}
{ add.period$
  write$
  newline$
}

%    \end{macrocode}
%
% \DescribeFunction{new.block}%
%   Declare family of functions to put punctuation marks  depending of current
%   status of output stack. The just check output state and revert it another
%   state if required. Checking output state prevents occasional doubling of
%   punctuation marks.
%    \begin{macrocode}
FUNCTION {new.block}
{ output.state before.all =
    'skip$
    { after.block 'output.state := }
  if$
}

%    \end{macrocode}
% \DescribeFunction{new.dblslash}%
%    \begin{macrocode}
FUNCTION {new.dblslash}
{ output.state before.all =
    'skip$
    { after.dblslash 'output.state := }
  if$
}

%    \end{macrocode}
% \DescribeFunction{new.slash}%
%    \begin{macrocode}
FUNCTION {new.slash}
{ output.state before.all =
    'skip$
    { after.slash 'output.state := }
  if$
}

%    \end{macrocode}
% \DescribeFunction{new.colon}%
%    \begin{macrocode}
FUNCTION {new.colon}
{ output.state before.all =
    'skip$
    { after.colon 'output.state := }
  if$
}

%    \end{macrocode}
% \DescribeFunction{new.semicolon}%
%    \begin{macrocode}
FUNCTION {new.semicolon}
{ output.state before.all =
    'skip$
    { after.semicolon 'output.state := }
  if$
}

%    \end{macrocode}
% \DescribeFunction{new.sentence}%
%    \begin{macrocode}
FUNCTION {new.sentence}
{ output.state after.block =
    'skip$
    { output.state before.all =
        'skip$
        { after.sentence 'output.state := }
      if$
    }
  if$
}

%    \end{macrocode}
% \DescribeFunction{add.blank}%
%    \begin{macrocode}
FUNCTION {add.blank}
{ " " * before.all 'output.state :=
}

%    \end{macrocode}
%
% \DescribeFunction{not}
%   Declare few logical functions.
%    \begin{macrocode}
FUNCTION {not}
{   { #0 }
    { #1 }
  if$
}

%    \end{macrocode}
% \DescribeFunction{and}
%    \begin{macrocode}
FUNCTION {and}
{   'skip$
    { pop$ #0 }
  if$
}

%    \end{macrocode}
% \DescribeFunction{or}
%    \begin{macrocode}
FUNCTION {or}
{   { pop$ #1 }
    'skip$
  if$
}

%    \end{macrocode}
% \DescribeFunction{chop.word}
% The function |chop.word| in |substr len str chop.word| removes given substring
% |substr| of length |len| from the beginning of the string |str|.
%    \begin{macrocode}
%<*sort|natbib>
INTEGERS { len }

FUNCTION {chop.word}
{ 's :=
  'len :=
  s #1 len substring$ =
    { s len #1 + global.max$ substring$ }
    's
  if$
}
%</sort|natbib>

%    \end{macrocode}
% \DescribeFunction{non.stop}
%    \begin{macrocode}
FUNCTION {non.stop}
{ duplicate$
   "}" * add.period$
   #-1 #1 substring$ "." =
}

%    \end{macrocode}
% \DescribeFunction{new.block.checka}
%    \begin{macrocode}
FUNCTION {new.block.checka}
{ empty$
    'skip$
    'new.block
  if$
}

%    \end{macrocode}
% \DescribeFunction{new.block.checkb}
%    \begin{macrocode}
FUNCTION {new.block.checkb}
{ empty$
  swap$ empty$
  and
    'skip$
    'new.block
  if$
}

%    \end{macrocode}
% \DescribeFunction{new.sentence.checka}
%    \begin{macrocode}
FUNCTION {new.sentence.checka}
{ empty$
    'skip$
    'new.sentence
  if$
}

%    \end{macrocode}
% \DescribeFunction{new.sentence.checkb}
%    \begin{macrocode}
FUNCTION {new.sentence.checkb}
{ empty$
  swap$ empty$
  and
    'skip$
    'new.sentence
  if$
}

%    \end{macrocode}
% \DescribeFunction{new.dblslash.checka}
% For |online| entry.
% \changes{v1.1}{15 Jan 2012}{entry type \texttt{@online} added}
%    \begin{macrocode}
FUNCTION {new.dblslash.checka}
{ empty$
    'skip$
    'new.dblslash
  if$
}

%    \end{macrocode}
% \DescribeFunction{field.or.null}
%    \begin{macrocode}
FUNCTION {field.or.null}
{ duplicate$ empty$
    { pop$ "" }
    'skip$
  if$
}

%    \end{macrocode}
%
% \DescribeFunction{emphasize}
% Declare function to emphasize last string in stack.
% \changes{v0.8}{10 Aug 2003}{macro \cs{BibEmph} added}
%    \begin{macrocode}
FUNCTION {emphasize}
{ duplicate$ empty$
    { pop$ "" }
    { "\BibEmph{" swap$ * "}" * }
  if$
}

%    \end{macrocode}
%
% \DescribeFunction{bracify}
% New in v.1.2. An idea borrowed from apsrev4-1.bst.
% Declare function to enclose last word by braces even if empty.
%    \begin{macrocode}
FUNCTION {bracify}
{ duplicate$ empty$
    { pop$ "{}" }
    { "{" swap$ * "}" * }
  if$
}

%    \end{macrocode}
% \DescribeFunction{enclose.square.brackets}
% This and the next functions are used to enclose last word by brackets.
%    \begin{macrocode}
FUNCTION {enclose.square.brackets}
{
  duplicate$ empty$
    { pop$ "" }
    { "[" swap$ * "]" * }
  if$
}

%    \end{macrocode}
% \DescribeFunction{enclose.round.brackets}
%    \begin{macrocode}
FUNCTION {enclose.round.brackets}
{
  duplicate$ empty$
    { pop$ "" }
    { "(" swap$ * ")" * }
  if$
}

%    \end{macrocode}
%
% \DescribeFunction{space.word}
% |space.word| inserts space before last string in stack.
%    \begin{macrocode}
FUNCTION {space.word}
{ " " swap$ * " " * }

%    \end{macrocode}
%
% \subsection{Standard abbreviations}
%
% Declare language-sensitive abbreviations.
% \DescribeFunction{bbl.edby}
%    \begin{macrocode}
FUNCTION {bbl.edby}    %  { "\bbledby{}" }
{ curlanguage "english" =
   {"Ed.\ by"}
   { curlanguage "ukrainian" =
%<!utf8>      {"{\cyr\CYRP\cyrii\cyrd\ \cyrr\cyre\cyrd.}"}
%<utf8>      {"Під ред."}
      { curlanguage "russian" =
%<!utf8>         {"{\cyr\CYRP\cyro\cyrd\ \cyrr\cyre\cyrd.}"}
%<utf8>         {"Под ред."}
         { curlanguage "german" =
            { "ed." }
            {"language is not defined: " language "edby" * * warning$ "Ed.\ by"}
         if$}
      if$}
   if$}
if$}

%    \end{macrocode}
% \DescribeFunction{bbl.cmplr}
%    \begin{macrocode}
FUNCTION {bbl.cmplr}
{ curlanguage "english" =
   { "Compiler"}
   { curlanguage "german" =
      { "Hrsg." }
      { curlanguage "ukrainian" =
%<!utf8>         {"{\cyr\CYRU\cyrk\cyrl.}"}
%<utf8>         {"{Укл.}"}
         { curlanguage "russian" =
%<!utf8>            {"{\cyr\CYRS\cyro\cyrs\cyrt.}"}
%<utf8>            {"{Сост.}"}
            {"language is not defined: " language  "cmplr" * * warning$ "Compiler"}
         if$}
      if$}
   if$}
if$}

%    \end{macrocode}
% \DescribeFunction{bbl.edition}
%    \begin{macrocode}
FUNCTION {bbl.edition} %  { "\bbledition{}" }
{ curlanguage "english" =
   {"edition"}
   { curlanguage "ukrainian" =
%<!utf8>      {"{\cyr\cyrv\cyri\cyrd.}"}
%<utf8>      {"{вид.}"}
      { curlanguage "russian" =
%<!utf8>         {"{\cyr\cyri\cyrz\cyrd.}"}
%<utf8>         {"{изд.}"}
         { curlanguage "german" =
            {" {aus.}" } %%%% { "Aufl." } ??
            { curlanguage "italian" =
               {"edizione"}
               { curlanguage "french" =
                  {"\'{e}dition"}
                  {"language is not defined: " language  "edition" * * warning$ "edition"}
               if$}
            if$}
         if$}
      if$}
   if$}
if$}

%    \end{macrocode}
% \DescribeFunction{bbl.vvolume}
%    \begin{macrocode}
FUNCTION {bbl.vvolume} %  { "\bblVolume{}" }
{ curlanguage "english" = curlanguage "french" = or curlanguage "italian" = or
   {"Volume"}
   { curlanguage "ukrainian" = curlanguage "russian" = or
%<!utf8>      { "\CYRT\cyro\cyrm" }
%<utf8>      { "Том" }
      { curlanguage "german" =
         {"{Band}"} %%%% { "Volumen" }
         {"language is not defined: " language  "vvolume" * * warning$ "Volume"}
      if$}
   if$}
if$}

%    \end{macrocode}
% \DescribeFunction{bbl.vvol}
%    \begin{macrocode}
FUNCTION {bbl.vvol}    %  { "\bblVol{}" }
{ curlanguage "english" = curlanguage "french" = or curlanguage "italian" = or
   {"Vol."}
   { curlanguage "ukrainian" = curlanguage "russian" = or
%<!utf8>      {"\CYRT."}
%<utf8>      {"Т."}
      { curlanguage "german" =
         {"{Bd.}"} %%%% { "Vol." }
         {"language is not defined: " language  "vvol" * * warning$ "Vol."}
      if$}
   if$}
if$}

%    \end{macrocode}
% \DescribeFunction{bbl.iissue}
%    \begin{macrocode}
FUNCTION {bbl.iissue}  %  { "\bblIssue{}" }
{ curlanguage "english" =
   {"Issue"}
   { curlanguage "ukrainian" =
%<!utf8>      {"\CYRV\cyri\cyrp\cyru\cyrs\cyrk"}
%<utf8>      {"Випуск"}
      { curlanguage "russian" =
%<!utf8>         {"\CYRV\cyrery\cyrp\cyru\cyrs\cyrk"}
%<utf8>         {"Выпуск"}
         { curlanguage "german" =
            {"{Heft}"} %%%% { "Ausgabe" }
            {"language is not defined: " language "iissue" * * warning$ "Issue"}
         if$}
      if$}
   if$}
if$}

%    \end{macrocode}
% \DescribeFunction{bbl.iiss}
%    \begin{macrocode}
FUNCTION {bbl.iiss}  %  { "\bblIss{}" }
{ curlanguage "english" =
   {"Iss."}
   { curlanguage "ukrainian" =
%<!utf8>      {"\CYRV\cyri\cyrp."}
%<utf8>      {"Вип."}
      { curlanguage "russian" =
%<!utf8>         {"\CYRV\cyrery\cyrp."}
%<utf8>         {"Вып."}
         { curlanguage "german" =
            {"{H.}"}
            {"language is not defined: " language "iiss" * * warning$ "Iss."}
         if$}
      if$}
   if$}
if$}

%    \end{macrocode}
% \DescribeFunction{bbl.of}
%    \begin{macrocode}
FUNCTION {bbl.of}      %  { "\bblof{}" }
{ curlanguage "english" =
   {"of"}
   { curlanguage "german" =
      { "von" }
      { curlanguage "ukrainian" =
%<!utf8>         { "{\cyr\cyrii\cyrz}" }
%<utf8>         { "{із}" }
         { curlanguage "russian" =
%<!utf8>            { "{\cyr\cyri\cyrz}" }
%<utf8>            { "{из}" }
            {"language is not defined: " language "of" * * warning$ "of"}
         if$}
      if$}
   if$}
if$}

%    \end{macrocode}
% \DescribeFunction{bbl.etal}
%    \begin{macrocode}
FUNCTION {bbl.etal}      %  { "\bblof{}" }
{ curlanguage "english" =
   {"et~al."}
   { curlanguage "german" =
      { "u.~a." }
      { curlanguage "ukrainian" =
%<!utf8>         {"{\cyr\cyrt\cyra~\cyrii\cyrn.}"}
%<utf8>         {"{та~ін.}"}
         { curlanguage "russian" =
%<!utf8>            {"{\cyr\cyri~\cyrd\cyrr.}"}
%<utf8>            {"и~др."}
            {"language is not defined: " language  "et~al" * * warning$ "et~al."}
         if$}
      if$}
   if$}
if$}

%    \end{macrocode}
% \DescribeFunction{bbl.and}
%    \begin{macrocode}
%<*natbib>
FUNCTION {bbl.and}      %  { "\bbland{}" }
{ curlanguage "english" =
   {"and"}
   { curlanguage "german" =
      { "und" }
      { curlanguage "ukrainian" =
%<!utf8>         {"{\cyrii}"}
%<utf8>         {"і"}
         { curlanguage "russian" =
%<!utf8>            {"{\cyri}"}
%<utf8>            {"и"}
            {"language is not defined: " language  "and" * * warning$ "and"}
         if$}
      if$}
   if$}
if$}
%</natbib>

%    \end{macrocode}
% \DescribeFunction{bbl.nnumber}
%    \begin{macrocode}
FUNCTION {bbl.nnumber} %  { "\bblNumber{}" }
{ curlanguage "english" =
   {"Number"}
   { curlanguage "ukrainian" = curlanguage "russian" = or
%<!utf8>      { "\CYRN\cyro\cyrm\cyre\cyrr" }
%<utf8>      { "{Номер}" }
      { curlanguage "german" =
         {"{Heft}"} %%% { "Anzahl" }
         {"language is not defined: " language  "nnumber" * * warning$ "Number"}
      if$}
   if$}
if$}

%    \end{macrocode}
% \DescribeFunction{bbl.number}
%    \begin{macrocode}
FUNCTION {bbl.number}  %  { "\bblnumber{}" }
{ curlanguage "english" =
   {"number"}
   { curlanguage "ukrainian" = curlanguage "russian" = or
%<!utf8>      {"{\cyr\cyrn\cyro\cyrm\cyre\cyrr}"}
%<utf8>      {"{номер}"}
      { curlanguage "german" =
         {"{Heft}"} %%% { "anzahl" }???
         {"language is not defined: " language  "number" * * warning$ "number"}
      if$}
   if$}
if$}

%    \end{macrocode}
% \DescribeFunction{bbl.nr}
%    \begin{macrocode}
FUNCTION {bbl.nr}     %   { "\bblno{}" }
{ curlanguage "english" =
   {"no."}
   { curlanguage "italian" =
      { "no" }
      { curlanguage "ukrainian" = curlanguage "russian" = or
%<!utf8>         { "{\cyr\textnumero}" }
%<utf8>         { "{№}" }
         { curlanguage "german" =
            {"{H.}"} %%% { "an." }
            {"language is not defined: " language "nr" * * warning$ "no."}
         if$}
      if$}
   if$}
if$}

%    \end{macrocode}
% \DescribeFunction{bbl.nnr}
%    \begin{macrocode}
FUNCTION {bbl.nnr}     %   { "\bblno{}" }
{ curlanguage "english" =
   {"No."}
   { curlanguage "ukrainian" = curlanguage "russian" = or
%<!utf8>      { "{\cyr\textnumero}" }
%<utf8>      { "{№}" }
      { curlanguage "german" =
         {"{H.}"} %%% { "an." }
         {"language is not defined: " language  "nnr" * * warning$ "No."}
      if$}
   if$}
if$}

%    \end{macrocode}
% \DescribeFunction{bbl.in}
%    \begin{macrocode}
FUNCTION {bbl.in}    %    { "\bblin{}" }
{ curlanguage "english" = curlanguage "german" = or
   {"in"}
   { curlanguage "ukrainian" = curlanguage "russian" = or
%<!utf8>      { "{\cyr\cyrv}" }
%<utf8>      { "{в}" }
      {"language is not defined: " language  "in" * * warning$ "in"}
   if$}
if$}

%    \end{macrocode}
% \DescribeFunction{bbl.iin}
%    \begin{macrocode}
FUNCTION {bbl.iin}  %     { "\bblIn{}" }
{ curlanguage "english" = curlanguage "german" = or
   {"In"}
   { curlanguage "ukrainian" = curlanguage "russian" = or
%<!utf8>      { "\CYRV" }
%<utf8>      { "{В}" }
      {"language is not defined: " language  "iin" * * warning$ "In"}
   if$}
if$}

%    \end{macrocode}
% \DescribeFunction{bbl.pages}
%    \begin{macrocode}
FUNCTION {bbl.pages}    % { "\bblpp." }
{ curlanguage "english" = curlanguage "french" = or curlanguage "italian" = or
   {"p."} %%% {"pp."}
   { curlanguage "ukrainian" = curlanguage "russian" = or
%<!utf8>      {"{\cyr\cyrs.}"}
%<utf8>      {"{с.}"}
      { curlanguage "german" =
         {"S."} %%%% { "s." }
         {"language is not defined: " language  "pages" * * warning$ "p."}
      if$}
   if$}
if$}

%    \end{macrocode}
% \DescribeFunction{bbl.page}
%    \begin{macrocode}
FUNCTION {bbl.page}   %   { "\bblp."
{ curlanguage "english" = curlanguage "french" = or curlanguage "italian" = or
   {"p."}
   { curlanguage "ukrainian" = curlanguage "russian" = or
%<!utf8>      {"{\cyr\cyrs.}"}
%<utf8>      {"{с.}"}
      { curlanguage "german" =
         {"S."} %%%% { "s." }
         {"language is not defined: " language  "page" * * warning$ "p."}
      if$}
   if$}
if$}

%    \end{macrocode}
% \DescribeFunction{bbl.ppages}
%    \begin{macrocode}
FUNCTION {bbl.ppages}%    { "\bblPp." }
{ curlanguage "english" = curlanguage "french" = or curlanguage "italian" = or
   {"P."} %%%% { "Pp." }
   { curlanguage "ukrainian" = curlanguage "russian" = or
%<!utf8>      {"{\cyr\CYRS.}"}
%<utf8>      {"{С.}"}
      { curlanguage "german" =
         {"S."}
         {"language is not defined: " language "ppages" * * warning$ "P."}
      if$}
   if$}
if$}

%    \end{macrocode}
% \DescribeFunction{bbl.ppage}
%    \begin{macrocode}
FUNCTION {bbl.ppage}  %   { "\bblP." }
{ curlanguage "english" = curlanguage "french" = or curlanguage "italian" = or
   {"P."}
   { curlanguage "ukrainian" = curlanguage "russian" = or
%<!utf8>      {"{\cyr\CYRS.}"}
%<utf8>      {"{С.}"}
      { curlanguage "german" =
         {"S."}
         {"language is not defined: " language  "ppage" * * warning$ "P."}
      if$}
   if$}
if$}

%    \end{macrocode}
% Next function was added in version 2016.07.07.
% \DescribeFunction{bbl.url}
% \changes{v1.2e}{07 Jul 2016}{\texttt{bbl.url} added to replace URL string}
%    \begin{macrocode}
FUNCTION {bbl.url}
{ curlanguage "english" =
   {"Access mode"}
   { curlanguage "ukrainian" =
%<!utf8>      { "{\CYRR\cyre\cyrzh\cyri\cyrm\ \cyrd\cyro\cyrs\cyrt\cyru\cyrp\cyru}" }
%<utf8>      { "{Режим доступу}" }
      { curlanguage "russian" =
%<!utf8>         { "{\CYRR\cyre\cyrzh\cyri\cyrm\ \cyrd\cyro\cyrs\cyrt\cyru\cyrp\cyra}" }
%<utf8>         { "{Режим доступа}" }
         { curlanguage "german" =
            { "{online; abgerufen}" }
            { "language is not defined: " language "urldate" * * warning$ "online; accessed" }
         if$}
      if$}
   if$}
if$}
%    \end{macrocode}
% Next function was added in version 2012.01.15.
% \DescribeFunction{bbl.urldate}
% \changes{v1.2}{13 Feb 2012}{fix \texttt{bbl.urldate} for ukrainian (Andrey Shvajkoy)}
%    \begin{macrocode}
FUNCTION {bbl.urldate}
{ curlanguage "english" =
   {"online; accessed"}
   { curlanguage "ukrainian" =
%<!utf8>      { "{\cyrd\cyra\cyrt\cyra\ \cyrz\cyrv\cyre\cyrr\cyrn\cyre\cyrn\cyrn\cyrya}" }
%<utf8>      { "{дата звернення}" }
      { curlanguage "russian" =
%<!utf8>         { "{\cyrd\cyra\cyrt\cyra\ \cyro\cyrb\cyrr\cyra\cyrshch\cyre\cyrn\cyri\cyrya}" }
%<utf8>         { "{дата обращения}" }
         { curlanguage "german" =
            { "{online; abgerufen}" }
            { "language is not defined: " language "urldate" * * warning$ "online; accessed" }
         if$}
      if$}
   if$}
if$}

%    \end{macrocode}
% \DescribeFunction{bbl.techreport}
%    \begin{macrocode}
FUNCTION {bbl.techreport} % rename to bbl.techreport
{ curlanguage "english" =
   { "Rep." }
   { curlanguage "german" =
      { "Bericht" }
      { curlanguage "russian" =
%<!utf8>         { "{\cyr\CYRO\cyrt\cyrch\cyre\cyrt}" }
%<utf8>         { "{Отчет}" }
         { "language is not defined: " language "techrep" * * warning$ "Rep." }
      if$}
   if$}
if$}

%    \end{macrocode}
% \DescribeFunction{bbl.mathesis}
%    \begin{macrocode}
FUNCTION {bbl.mathesis}
{ curlanguage "english" =
   { "Master's thesis" }
   { curlanguage "german" =
      { "Diss.~Mag." }
      { curlanguage "russian" =
%<!utf8>        { "{\cyr\CYRK\cyrv\cyra\cyrl\cyri\cyrf\cyri\cyrk\cyra\cyrc\cyri"
%<!utf8>          "\cyro\cyrn\cyrn\cyra\cyrya\ \cyrr\cyra\cyrb\cyro\cyrt\cyra\ " *
%<!utf8>          "\cyrm\cyra\cyrg\cyri\cyrs\cyrt\cyrr\cyra}" * }
%<utf8>         { "{Квалификационная работа магистра}" }
         { "language is not defined: " language "mthesis" * * warning$ "Master's thesis" }
      if$}
   if$}
if$}

%    \end{macrocode}
% \DescribeFunction{bbl.phdthesis}
% \changes{v1.2i}{04 Jan 2017}{corrected bbl.phdthesis and bbl.docthesis to comply GOST}
%    \begin{macrocode}
FUNCTION {bbl.phdthesis}
{ curlanguage "english" =
   { "Ph.\,D. thesis" }
   { curlanguage "german" =
      { "Diss.~Ph.\,D." }
      { curlanguage "russian" =
%<!utf8>         { "{\cyr\cyrd\cyri\cyrs.\ \ldots\ \cyrk\cyra\cyrn\cyrd. "
%<!utf8>           "\cyrn\cyra\cyru\cyrk}" * }
%<utf8>         { "{дис.\ \ldots\ канд. наук}" }
         { curlanguage "french" =
           { "Th\`{e}se de doctorat" }
           { "language is not defined: " language "phdthesis" * * warning$ "Ph.\,D. thesis" }
         if$}
      if$}
   if$}
if$}

%    \end{macrocode}
% \DescribeFunction{bbl.docthesis}
%    \begin{macrocode}
FUNCTION {bbl.docthesis}
{ curlanguage "english" =
   { "Dr.\,Sci. dissertation" }
   { curlanguage "german" =
      { "Diss.~Dr." }
      { curlanguage "russian" =
%<!utf8>         { "{\cyr\cyrd\cyri\cyrs.\ \ldots\ \cyrd-\cyrr\cyra\ "
%<!utf8>           "\cyrn\cyra\cyru\cyrk}" * }
%<utf8>         { "{дис.\ \ldots\ д-ра наук}" }
         { "language is not defined: " language "docthesis" * * warning$ "Dr.\,Sci. dissertation" }
      if$}
   if$}
if$}

%    \end{macrocode}
% \DescribeFunction{bbl.thesis.type`}
%    \begin{macrocode}
FUNCTION {bbl.thesis.type}
{ type "mathesis" =
  { bbl.mathesis }
  { type "phdthesis" =
    { bbl.phdthesis }
    { type "docthesis" =
      { bbl.docthesis }
      %%{ "!!!" type * "t" change.case$ }
%<!utf8>      { type "t" change.case$ }
%<utf8>      { type }
    if$}
  if$}
if$}

%    \end{macrocode}
% \DescribeFunction{bbl.nnoaddress}
%    \begin{macrocode}
FUNCTION {bbl.nnoaddress}
{ curlanguage "english" =
   { "S.\ l." }
   { curlanguage "russian" =
%<!utf8>      { "{\cyr\CYRB.\ \cyrm.}" }
%<utf8>      { "{Б.\ м.}" }
      { "language is not defined: " language "nnoaddress" * * warning$ "S.\ l." }
   if$}
if$}

%    \end{macrocode}
% \DescribeFunction{bbl.nopublisher}
%    \begin{macrocode}
FUNCTION {bbl.nopublisher}
{ curlanguage "english" =
   { "s.\ n." }
   { curlanguage "russian" =
%<!utf8>      { "{\cyr\cyrb.\ \cyri.}" }
%<utf8>      { "{б.\ и.}" }
      { "language is not defined: " language "nnopublisher" * * warning$ "s.\ n." }
   if$}
if$}

%    \end{macrocode}
% \DescribeFunction{bbl.nnopublisher}
%    \begin{macrocode}
FUNCTION {bbl.nnopublisher}
{ curlanguage "english" =
   { "S.\ n." }
   { curlanguage "russian" =
%<!utf8>      { "{\cyr\CYRB.\ \cyri.}" }
%<utf8>      { "{Б.\ и.}" }
      { "language is not defined: " language "nnopublisher" * * warning$ "S.\ n." }
   if$}
if$}

%    \end{macrocode}
% \DescribeFunction{bbl.media.text}
%    \begin{macrocode}
FUNCTION {bbl.media.text}
{ curlanguage "english" =
   { "Text" }
   { curlanguage "russian" = curlanguage "ukrainian" = or
%<!utf8>      { "{\cyr\CYRT\cyre\cyrk\cyrs\cyrt}" }
%<utf8>      { "{Текст}" }
      { "language is not defined: " language "media" * * warning$ "Text" }
   if$}
if$}

%    \end{macrocode}
% \DescribeFunction{bbl.media.eresource}
%    \begin{macrocode}
FUNCTION {bbl.media.eresource}
{ curlanguage "english" =
   { "Electronic resource" }
   { curlanguage "russian" =
%<!utf8>      { "{\cyr\CYREREV\cyrl\cyre\cyrk\cyrt\cyrr\cyro\cyrn\cyrn\cyrery\cyrishrt\ "
%<!utf8>        "\cyrr\cyre\cyrs\cyru\cyrr\cyrs}" * }
%<utf8>      { "{Электронный ресурс}" }
      { curlanguage "ukrainian" =
%<!utf8>        { "{\cyr\CYRE\cyrl\cyre\cyrk\cyrt\cyrr\cyro\cyrn\cyrn\cyri\cyrishrt\ "
%<!utf8>          "\cyrr\cyre\cyrs\cyru\cyrr\cyrs}" * }
%<utf8>        { "{Електронний ресурс}" }
        { "language is not defined: " language "media" * * warning$ "Electronic resource" }
      if$}
   if$}
if$}

%    \end{macrocode}
% \DescribeFunction{bbl.media.online}
% \changes{v1.2i}{02 Jan 2017}{\texttt{media} field normalised}
%    \begin{macrocode}
FUNCTION {bbl.media.online}
{ curlanguage "english" =
   { "Electronic resource online" }
   { curlanguage "russian" =
%<!utf8>      { "{\cyr\CYREREV\cyrl\cyre\cyrk\cyrt\cyrr\cyro\cyrn\cyrn\cyrery\cyrishrt\ "
%<!utf8>        "\cyrr\cyre\cyrs\cyru\cyrr\cyrs\ \cyro\cyrn\cyrl\cyra\cyrishrt\cyrn}" * }
%<utf8>      { "{Электронный ресурс онлайн}" }
      { curlanguage "ukrainian" =
%<!utf8>        { "{\cyr\CYRE\cyrl\cyre\cyrk\cyrt\cyrr\cyro\cyrn\cyrn\cyri\cyrishrt\ "
%<!utf8>          "\cyrr\cyre\cyrs\cyru\cyrr\cyrs\ \cyro\cyrn\cyrl\cyra\cyrishrt\cyr}" * }
%<utf8>        { "{Електронний ресурс онлайн}" }
        { "language is not defined: " language "media" * * warning$ "Electronic resource" }
      if$}
   if$}
if$}

%    \end{macrocode}
% \DescribeFunction{bbl.chief}
%    \begin{macrocode}
FUNCTION {bbl.chief}
{ curlanguage "english" =
   { "chief" }
   { curlanguage "russian" =
%<!utf8>      { "\cyrr\cyru\cyrk." }
%<utf8>      { "{рук.}" }
      { curlanguage "ukrainian" =
%<!utf8>        { "\cyrr\cyru\cyrk." }
%<utf8>        { "{рук.}" }
        { "language is not defined: " language "chief" * * warning$ "chief" }
      if$}
   if$}
if$}

%    \end{macrocode}
% \DescribeFunction{bbl.executor}
%    \begin{macrocode}
FUNCTION {bbl.executor}
{ curlanguage "english" =
   { "Executor" }
   { curlanguage "russian" =
%<!utf8>      { "{\cyr\cyri\cyrs\cyrp\cyro\cyrl\cyrn.}" }
%<utf8>      { "{исполн.}" }
      { curlanguage "ukrainian" =
%<!utf8>        { "{\cyr\cyrv\cyri\cyrk\cyro\cyrn\cyra\cyrv\cyre\cyrc\cyrsftsn}" }
%<utf8>        { "{виконавець}" }
        { "language is not defined: " language "executor" * * warning$ "executor" }
      if$}
   if$}
if$}

%    \end{macrocode}
% \DescribeFunction{bbl.media}
%    \begin{macrocode}
%FUNCTION {bbl.media}
%{ type$ "online" =
%   { bbl.media.eresource }
%   { bbl.media.text  }
%if$}
FUNCTION {bbl.media}
{ media "online" =
   { bbl.media.online }
   { media "eresource" =
     { bbl.media.eresource }
     { bbl.media.text }
   if$}
if$}

%    \end{macrocode}
% \DescribeFunction{bbl.req}
%    \begin{macrocode}
FUNCTION {bbl.req}
{
  curlanguage "english" =
    { "req." }
    { curlanguage "german" =
      { "ang." }
      { curlanguage "russian" =
%<!utf8>        { "{\cyr\cyrz\cyra\cyrya\cyrv\cyrl.}" }
%<utf8>        { "{заявл.}" }
        { "language is not defined: " language "req" * * warning$ "req" }
      if$
      }
    if$
    }
  if$
}

%    \end{macrocode}
% \DescribeFunction{bbl.publ}
%    \begin{macrocode}
FUNCTION {bbl.publ}
{
  curlanguage "english" =
    { "publ." }
    { curlanguage "german" =
      { "ausg." }
      { curlanguage "russian" =
%<!utf8>        { "{\cyr\cyro\cyrp\cyru\cyrb\cyrl.}" }
%<utf8>        { "{опубл.}" }
        { "language is not defined: " language "publication" * * warning$ "publication" }
      if$
      }
    if$
    }
  if$
}

%    \end{macrocode}
% \DescribeFunction{bbl.priority}
%    \begin{macrocode}
FUNCTION {bbl.priority}
{
  curlanguage "english" =
    { "priority" }
    { curlanguage "german" =
      { "Prioritat" }
      { curlanguage "russian" =
%<!utf8>        { "{\cyr\cyrp\cyrr\cyri\cyro\cyrr\cyri\cyrt\cyre\cyrt}" }
%<utf8>        { "{приоритет}" }
        { "language is not defined: " language "priority" * * warning$ "priority" }
      if$
      }
    if$
    }
  if$
}

%    \end{macrocode}
%
% \subsection{Formatting functions}
%
% Declare functions to format separate elements of bibliographic reference.
%    \begin{macrocode}
INTEGERS { nameptr namesleft numnames }


%    \end{macrocode}
%
% \DescribeFunction{format.names}
% Function |format.names| has 2 version.
% First is for bibliographic records rather then for bibliographic references.
% It is used for \file{.bst} styles compiled without the |modern| option.
% It format every name as `LastName, F. S.'.
% Historically, this version was used first for earlier styles included into
% GOST bundle.
%
% \textbf{Important note}
%
% Neither \file{bibtex} nor \file{bibtex8} can handle unicoded text without
% troubles. In particular, they fail to reduce a Cyrillic name to initials.
% Therefore we avoid using |f.| primitive when option |utf8| is in effect.

%    \begin{macrocode}
%<*!modern>
FUNCTION {format.names}
{
%<!utf8>  #1 "{vv~}{ll}{~jj}{,~f.}" format.name$
%<utf8>  #1 "{vv~}{ll}{~jj}{,~ff}" format.name$
}
%</!modern>
%    \end{macrocode}
% Second version drops comma from output so that every name is formatted
% as `LastName F. S.'. It also substitutes 4th and following names by
% localized term `et al'.
%    \begin{macrocode}
%<*modern>
FUNCTION {format.names}
{
  's :=
  #1 'nameptr :=
  s num.names$ 'numnames :=
  numnames 'namesleft :=
    { namesleft #0 > }
    { s nameptr
%<!utf8>      "{vv~}{ll}{~jj}{~f.}" format.name$ 't :=
%<utf8>      "{vv~}{ll}{~jj}{~ff}" format.name$ 't :=
      nameptr #1 >
        { nameptr #4 =
          numnames #4 > and
            { "others" 't :=
              #1 'namesleft := }
            'skip$
          if$
          namesleft #1 >
            { ", " * t * }
            { t "others" =
	          t "~others" =
	          or
%<!strict>                { " " * bbl.etal * }
%<strict>                { " " * bbl.etal enclose.square.brackets *}
                { ", " * t * }
              if$
            }
          if$
        }
        't
      if$
      nameptr #1 + 'nameptr :=
      namesleft #1 - 'namesleft :=
    }
  while$
}
%</modern>

%    \end{macrocode}
% \DescribeFunction{format.lab.names}
% Declare function to go to optional argument of |\bibitem| in
% the styles generated with the option |natbib|.
%    \begin{macrocode}
%<*natbib>
FUNCTION {format.lab.names}
{ 's :=
  language empty$
    { "english" 'curlanguage := }
    { language  'curlanguage := }
  if$
  s #1 "{vv~}{ll}" format.name$
  s num.names$ duplicate$
  #2 >
    %{ pop$ " et~al." * }
    { pop$ " " bbl.etal * * }
    { #2 <
        'skip$
        { s #2 "{ff }{vv }{ll}{ jj}" format.name$ "others" =
            %{ " et~al." * }
            { " " bbl.etal * * }
            %{ " and " * s #2 "{vv~}{ll}" format.name$ * }
            { " " bbl.and " " * * * s #2 "{vv~}{ll}" format.name$ * }
          if$
        }
      if$
    }
  if$
}
%</natbib>

%    \end{macrocode}
% \DescribeFunction{format.names.rev}
% Declare function to format names for authors/bookauthors list after title and etc.
% Note that |format.names.rev| cuts list of names to at most 4 persons.
% We do not cut names to initials in this list.
%    \begin{macrocode}
FUNCTION {format.names.rev}
{
  's :=
  #1 'nameptr :=
  s num.names$ 'numnames :=
  numnames 'namesleft :=
    { namesleft #0 > }
    { s nameptr
%<!utf8>      %"{f.}{~vv}{~ll}{, jj}" format.name$ 't :=
%<!utf8>      "{ff}{~vv}{~ll}{, jj}" format.name$ 't :=
%<utf8>      "{ff}{~vv}{~ll}{, jj}" format.name$ 't :=
      nameptr #1 >
        { nameptr #4 =
          numnames #4 > and
            { "others" 't :=
              #1 'namesleft := }
            'skip$
          if$
          namesleft #1 >
            { ", " * t * }
            { t "others" =
	          t "~others" =
	          or
%<!strict>                { " " * bbl.etal * }
%<strict>                { " " * bbl.etal enclose.square.brackets * }
                { ", " * t * }
              if$
            }
          if$
        }
        't
      if$
      nameptr #1 + 'nameptr :=
      namesleft #1 - 'namesleft :=
    }
  while$
}

%    \end{macrocode}
% \DescribeFunction{format.key}
% Function to substitute empty field (usually, author name) with the |key| field.
%    \begin{macrocode}
%<*natbib>
FUNCTION {format.key}
{ empty$
    { key field.or.null }
    { "" }
  if$
}
%</natbib>

%    \end{macrocode}
% \DescribeFunction{format.authors}
%    \begin{macrocode}
FUNCTION {format.authors}
{ author empty$
    { "" }
    { author format.names emphasize}
  if$
}

%    \end{macrocode}
% \DescribeFunction{author.key.label}
%    \begin{macrocode}
%<*natbib>
FUNCTION {author.key.label}
{ author empty$
    { key empty$
        { cite$ #1 #3 substring$ }
        'key
      if$
    }
    { author format.lab.names }
  if$
}

FUNCTION {author.editor.key.label}
{ author empty$
    { editor empty$
        { key empty$
            { cite$ #1 #3 substring$ }
            %'key          %% causes lost of year
            { "{}" key * } %% Bug in bibtex8 ??
          if$
        }
        { editor format.lab.names }
      if$
    }
    { author format.lab.names }
  if$
}

FUNCTION {author.key.organization.label}
{ author empty$
    { key empty$
        { organization empty$
            { cite$ #1 #3 substring$ }
            { "The " #4 organization chop.word #3 text.prefix$ }
          if$
        }
        'key
      if$
    }
    { author format.lab.names }
  if$
}

FUNCTION {editor.key.organization.label}
{ editor empty$
    { key empty$
        { organization empty$
            { cite$ #1 #3 substring$ }
            { "The " #4 organization chop.word #3 text.prefix$ }
          if$
        }
        'key
      if$
    }
    { editor format.lab.names }
  if$
}

FUNCTION {calc.short.authors}
{ type$ "book" =
  type$ "inbook" =
  or
    'author.editor.key.label
    { type$ "proceedings" =
        'editor.key.organization.label
        { type$ "manual" =
            'author.key.organization.label
            'author.key.label
          if$
        }
      if$
    }
  if$
  'short.list :=
}

FUNCTION {calc.label}
{ calc.short.authors
  short.list
  "("
  *
  year duplicate$ empty$
  short.list key field.or.null = or
     { pop$ "" }
     'skip$
  if$
  *
  'label :=
}

%</natbib>

%    \end{macrocode}
% \DescribeFunction{format.bookauthors}
% This function is used only once, in |bookauthor.before|, and the latter is
% used only in |inbook| entry.
%    \begin{macrocode}
FUNCTION {format.bookauthors}
{ bookauthor empty$
    { "" }
    { bookauthor format.names}
  if$
}

%    \end{macrocode}
% \DescribeFunction{format.authors.after}
%    \begin{macrocode}
FUNCTION {format.authors.after}
{ author empty$
    { "" }
    { author format.names.rev}
  if$
}

%    \end{macrocode}
% \DescribeFunction{format.bookauthors.after}
%    \begin{macrocode}
FUNCTION {format.bookauthors.after}
{ bookauthor empty$
    { "" }
    { bookauthor format.names.rev}% always cuts to 4 persons
  if$
}

%    \end{macrocode}
% \DescribeFunction{format.editors.after}
%    \begin{macrocode}
FUNCTION {format.editors.after}
{ editor empty$
    { "" }
    { bbl.edby "\ " * editor format.names.rev * }
  if$
}

%    \end{macrocode}
% \DescribeFunction{format.chief.after}
%    \begin{macrocode}
FUNCTION {format.chief.after}
{ editor empty$
    { "" }
    { bbl.chief "\ " * editor format.names.rev * }
  if$
}

%    \end{macrocode}
% \DescribeFunction{format.executor.after}
%    \begin{macrocode}
FUNCTION {format.executor.after}
{ author empty$
    { "" }
    { bbl.executor ": " * author format.names.rev * }
  if$
}

%    \end{macrocode}
% \DescribeFunction{format.compiler.after}
%    \begin{macrocode}
FUNCTION {format.compiler.after}
{ compiler empty$
    { "" }
    { bbl.cmplr "\ " * compiler format.names.rev * }
  if$
}

%    \end{macrocode}
% \DescribeFunction{format.title}
% \textbf{Important note}
% \
% Neither \file{bibtex} nor \file{bibtex8} can handle unicoded text without
% troubles. In particular, \file{bibtex8} fails to change case of a string if it
% contains Cyrillic letter. Therefore we avoid using |change.case$| when
% option |utf8| is in effect.
%    \begin{macrocode}
FUNCTION {format.title}
{ title empty$
    { "" }
%<!utf8>    { title "t" change.case$ }
%<utf8>    { title }
  if$
}

%    \end{macrocode}
% \DescribeFunction{format.month}
% New in version 1.2d. This macro reads month field and translate English names
% of months to Russian if current language is Russian.
% \changes{v1.2i}{29 Dec 2016}{typo fix in \texttt{format.month}}
%    \begin{macrocode}
FUNCTION {format.month}
{ month empty$
    { "" }
    { curlanguage "russian" =
        { month "Jan." =
%<!utf8>            { "\CYRYA\cyrn\cyrv." }
%<utf8>            { "Янв." }
        { month "Feb." =
%<!utf8>            { "\CYRF\cyre\cyrv\cyrr." }
%<utf8>            { "Фев." }
        { month "Mar." =
%<!utf8>            { "\CYRM\cyra\cyrr\cyrt" }
%<utf8>            { "Март" }
        { month "Apr." =
%<!utf8>            { "\CYRA\cyrp\cyrr." }
%<utf8>            { "Апр." }
        { month "May" =
%<!utf8>            { "\CYRM\cyra\cyrishrt" }
%<utf8>            { "Май" }
        { month "Jun." =
%<!utf8>            { "\CYRI\cyryu\cyrn\cyrsftsn" }
%<utf8>            { "Июнь" }
        { month "Jul." =
%<!utf8>            { "\CYRI\cyryu\cyrl\cyrsftsn" }
%<utf8>            { "Июль" }
        { month "Aug." =
%<!utf8>            { "\CYRA\cyrv\cyrg\." }
%<utf8>            { "Авг." }
        { month "Sep." =
%<!utf8>            { "\CYRS\cyre\cyrn\cyrt." }
%<utf8>            { "Сент." }
        { month "Oct." =
%<!utf8>            { "\CYRO\cyrk\cyrt." }
%<utf8>            { "Окт." }
        { month "Nov." =
%<!utf8>            { "\CYRN\cyro\cyrya\cyrb." }
%<utf8>            { "Нояб." }
        { month "Dec." =
%<!utf8>            { "\CYRD\cyre\cyrk." }
%<utf8>            { "Дек." }
            { month }
            if$}if$}if$}if$}if$}if$}if$}if$}if$}if$}if$}if$}
        { month }
        if$}
    if$
}
%    \end{macrocode}
% \DescribeFunction{format.date}
% \changes{v1.2c}{17 Jan 2015}{fix Gost2003: \texttt{"}\texttt{-}\texttt{-}\texttt{-} replaced by macro \cs{BibDash}}
%    \begin{macrocode}
FUNCTION {format.date}
{ year empty$
    { month empty$
        { "" }
        { "there's a month but no year in " cite$ * warning$
          format.month
        }
      if$
    }
    { month empty$
        'year
        { year ". \BibDash " format.month * * }
      if$
    }
  if$
%<natbib>  extra.label * % new in v.1.2
}

%    \end{macrocode}
% \DescribeFunction{address.or.location}
% New function in version 1.2i. Returns |address| value if not empty; otherwise
% returns |location| value even if the latter is empty.
% \changes{v1.2i}{02 Jan 2017}{\texttt{location} field added as alias to \texttt{address} field}
%    \begin{macrocode}
%FUNCTION {address.or.location}
%{
%  address empty$
%    { location empty$
%      'skip$
%      { location }
%    if$ }
%    { address }
%  if$
%}
FUNCTION {address.or.location}
{
  address empty$
    { location }
    { address }
  if$
}

%    \end{macrocode}
%
% \DescribeFunction{specialitycode.or.number}
% New function in version 1.2i. Returns |specialitycode| value if not empty; otherwise
% returns |number| value even if the latter is empty.
% \changes{v1.2i}{13 Jan 2017}{\texttt{number} field is now alias to \texttt{specialitycode}}
%    \begin{macrocode}
FUNCTION {specialitycode.or.number}
{
  specialitycode empty$
    { number }
    { specialitycode }
  if$
}

%    \end{macrocode}
%
% \DescribeFunction{institution.or.school}
% New function in version 1.2i. Returns |institution| value if not empty; otherwise
% returns |school| value even if the latter is empty.
% \changes{v1.2i}{13 Jan 2017}{\texttt{school} field is now alias to \texttt{instituttion}}
%    \begin{macrocode}
FUNCTION {institution.or.school}
{
  institution empty$
    { school}
    { institution }
  if$
}
%    \end{macrocode}
%
% \DescribeFunction{format.address.publisher}
% \DescribeFunction{format.address.publisher.date}
%    \begin{macrocode}
%<*!strict>
%FUNCTION {output.address.publisher}
%{
%  address empty$
%    'skip$
%    { address output
%      publisher empty$
%        'skip$
%        { new.colon }
%      if$
%    }
%  if$
%  publisher output
%}
FUNCTION {output.address.publisher}
{
  address empty$ location empty$ and
    'skip$
    { address.or.location output
      publisher empty$
        'skip$
        { new.colon }
      if$
    }
  if$
  publisher output
}
%</!strict>
%<*strict>
FUNCTION {output.address.publisher}
{
  address empty$
    {
      bbl.nnoaddress
      publisher empty$
        { "~: " * bbl.nopublisher * enclose.square.brackets }
        { enclose.square.brackets "~: " * publisher * }
      if$
    }
    {
      address output
      new.colon
      publisher empty$
        { bbl.nopublisher enclose.square.brackets }
        { publisher }
      if$
    }
  if$
  output
}

%</strict>

%    \end{macrocode}
% \DescribeFunction{output.address.publisher.date}
% |Otput.address.publisher.date| is used in old styles.
% New styles use |output.address.publisher|.
%    \begin{macrocode}
%<*!(modern|strict)>
FUNCTION {output.address.publisher.date}
{
  output.address.publisher
  format.date output
}
%</!(modern|strict)>

%    \end{macrocode}
% \DescribeFunction{output.bibitem}
% \changes{v1.2c}{18 Jan 2015}{entry field \texttt{langid} added}
%    \begin{macrocode}
%<*!natbib>
FUNCTION {output.bibitem}
{ newline$
  "\bibitem" write$
  cite$ bracify write$
  newline$
  langid empty$
    { language empty$
        { "english" 'curlanguage := }
        { language  'curlanguage := }
      if$
    }
    { langid  'curlanguage := }
  if$
  "\selectlanguageifdefined" curlanguage bracify * write$
  newline$
  ""
  before.all 'output.state :=
}

%</!natbib>
%    \end{macrocode}
% \DescribeFunction{format.full.names}
% \DescribeFunction{author.full}
% \DescribeFunction{editor.full}
% \DescribeFunction{author.editor.full}
% \DescribeFunction{make.full.names}
% \DescribeFunction{output.bibitem}
% In case of |natbib| option, we need |make.full.names| to compose
% |output.bibitem|, and the latter in its turn requires some more functions.
%    \begin{macrocode}
%<*natbib>
FUNCTION {format.full.names}
{'s :=
  language empty$
    { "english" 'curlanguage := }
    { language  'curlanguage := }
  if$
  #1 'nameptr :=
  s num.names$ 'numnames :=
  numnames 'namesleft :=
    { namesleft #0 > }
    { s nameptr
      "{vv~}{ll}" format.name$ 't :=
      nameptr #1 >
        {
          namesleft #1 >
            { ", " * t * }
            {
              numnames #2 >
              curlanguage "english" =
              and
                { "," * }
                'skip$
              if$
              t "others" =
	          %t "~others" =
	          %or
                %{ " et~al." * }
                { " " bbl.etal * * }
                %{ " and " * t * }
                { " " bbl.and " " * * * t * }
              if$
            }
          if$
        }
        't
      if$
      nameptr #1 + 'nameptr :=
      namesleft #1 - 'namesleft :=
    }
  while$
}

FUNCTION {author.full}
{ author empty$
    { "" }
    { author format.full.names }
  if$
}

FUNCTION {editor.full}
{ editor empty$
    { "" }
    { editor format.full.names }
  if$
}

FUNCTION {author.editor.full}
{ author empty$
    { editor empty$
        { "" }
        { editor format.full.names }
      if$
    }
    { author format.full.names }
  if$
}

FUNCTION {make.full.names}
{ type$ "book" =
  type$ "inbook" =
  or
    'author.editor.full
    { type$ "proceedings" =
        'editor.full
        'author.full
      if$
    }
  if$
}

% =======================================
FUNCTION {output.bibitem}
{ newline$
  "\bibitem[" write$
  label write$
  ")" make.full.names duplicate$ short.list =
     { pop$ }
     { * }
   if$
  "]{" * write$
  cite$ write$
  "}" write$
%%  language empty$
%%    { "english" 'curlanguage := }
%%    {language  'curlanguage := }
%%  if$
  langid empty$
    { language empty$
        { "english" 'curlanguage := }
        { language  'curlanguage := }
      if$
    }
    { langid  'curlanguage := }
  if$
  "\selectlanguageifdefined" curlanguage bracify * write$
  newline$
  ""
  before.all 'output.state :=
}
% =======================================
%FUNCTION {output.bibitem}
%{ newline$
%  "\bibitem" write$
%%  author.key.label
%%  year parenthesify *
%%  "; lbl:" label * *
%%  "; mfn:"  make.full.names * *
%  label
%  make.full.names *
%  bracketise write$
%  cite$ bracify write$
%  newline$
%  language empty$
%    { "english" 'curlanguage := }
%    {language  'curlanguage := }
%  if$
%  "\selectlanguageifdefined" curlanguage bracify * write$
%  newline$
%  ""
%  before.all 'output.state :=
%}
% =======================================
%</natbib>

%    \end{macrocode}
% \DescribeFunction{n.dashify}
%    \begin{macrocode}
FUNCTION {n.dashify}
{ 't :=
  ""
    { t empty$ not }
    { t #1 #1 substring$ "-" =
        { t #1 #2 substring$ "--" = not
            { "--" *
              t #2 global.max$ substring$ 't :=
            }
            {   { t #1 #1 substring$ "-" = }
                { "-" *
                  t #2 global.max$ substring$ 't :=
                }
              while$
            }
          if$
        }
        { t #1 #1 substring$ *
          t #2 global.max$ substring$ 't :=
        }
      if$
    }
  while$
}

%    \end{macrocode}
% \DescribeFunction{word.in}
%    \begin{macrocode}
FUNCTION {word.in}
{ bbl.iin
  " " * }

%    \end{macrocode}
% \DescribeFunction{format.btitle}
%    \begin{macrocode}
FUNCTION {format.btitle}
{ title
}

%    \end{macrocode}
% \DescribeFunction{tie.or.space.connect}
%    \begin{macrocode}
FUNCTION {tie.or.space.connect}
{ duplicate$ text.length$ #3 <
    { "~" }
    { " " }
  if$
  swap$ * *
}

%    \end{macrocode}
% \DescribeFunction{tie.connect}
% Declare function to insert unbreakable space before last word in stack.
%    \begin{macrocode}
FUNCTION {tie.connect}
 {"~"
  swap$ * *
 }


%    \end{macrocode}
% \DescribeFunction{either.or.chec}
%    \begin{macrocode}
FUNCTION {either.or.check}
{ empty$
    'pop$
    { "can't use both " swap$ * " fields in " * cite$ * warning$ }
  if$
}

%    \end{macrocode}
% \DescribeFunction{format.bvolume}
%    \begin{macrocode}
FUNCTION {format.bvolume}
{ volume empty$
    { "" }
    { bbl.vvol volume tie.connect
      series empty$
        'skip$
        { bbl.of space.word * series emphasize * }
      if$
      "volume and number" number either.or.check
    }
  if$
}

%    \end{macrocode}
% \DescribeFunction{format.number.series}
%    \begin{macrocode}
FUNCTION {format.number.series}
{ volume empty$
    { number empty$
        { series field.or.null }
        { series empty$
            { "there's a number but no series in " cite$ * warning$
              bbl.nnr }
            {
              %new.dblslash
              new.sentence
              series
              bbl.nr
              tie.or.space.connect}
          if$
          number tie.or.space.connect
        }
      if$
    }
    { "" }
  if$
}

%    \end{macrocode}
% \DescribeFunction{is.num}
%    \begin{macrocode}
FUNCTION {is.num}
{ chr.to.int$
  duplicate$ "0" chr.to.int$ < not
  swap$ "9" chr.to.int$ > not and
}

%    \end{macrocode}
% \DescribeFunction{extract.num}
%    \begin{macrocode}
FUNCTION {extract.num}
{ duplicate$ 't :=
  "" 's :=
  { t empty$ not }
  { t #1 #1 substring$
    t #2 global.max$ substring$ 't :=
    duplicate$ is.num
      { s swap$ * 's := }
      { pop$ "" 't := }
    if$
  }
  while$
  s empty$
    'skip$
    { pop$ s }
  if$
}

%<*debug>
%    \end{macrocode}
% \DescribeFunction{eng.ord}
%    \begin{macrocode}
FUNCTION {eng.ord}
{ duplicate$ "1" swap$ *
  #-2 #1 substring$ "1" =
     { bbl.th * }
     { duplicate$ #-1 #1 substring$
       duplicate$ "1" =
         { pop$ bbl.st * }
         { duplicate$ "2" =
             { pop$ bbl.nd * }
             { "3" =
                 { bbl.rd * }
                 { bbl.th * }
               if$
             }
           if$
          }
       if$
     }
   if$
}
%</debug>

%    \end{macrocode}
% \DescribeFunction{convert.edition}
%    \begin{macrocode}
FUNCTION {convert.edition}
{ edition
% edition extract.num "l" change.case$ 's :=
%  s "first" = s "1" = or
%    { bbl.first 't := }
%    { s "second" = s "2" = or
%        { bbl.second 't := }
%        { s "third" = s "3" = or
%            { bbl.third 't := }
%            { s "fourth" = s "4" = or
%                { bbl.fourth 't := }
%                { s "fifth" = s "5" = or
%                    { bbl.fifth 't := }
%                    { s #1 #1 substring$ is.num
%                        { s eng.ord 't := }
%                        { edition 't := }
%                      if$
%                    }
%                  if$
%                }
%              if$
%            }
%          if$
%        }
%      if$
%    }
%  if$
%  t
}

%    \end{macrocode}
% \DescribeFunction{format.edition}
%    \begin{macrocode}
FUNCTION {format.edition}
{ edition empty$
    { "" }
    { output.state mid.sentence =
%<!utf8>        { convert.edition "l" change.case$ " " * bbl.edition * }
%<!utf8>        { convert.edition "t" change.case$ " " * bbl.edition * }
%<utf8>        { convert.edition " " * bbl.edition * }
%<utf8>        { convert.edition " " * bbl.edition * }
      if$
    }
  if$
}

INTEGERS { multiresult }

%    \end{macrocode}
% \DescribeFunction{multi.page.check}
%    \begin{macrocode}
FUNCTION {multi.page.check}
{ 't :=
  #0 'multiresult :=
    { multiresult not
      t empty$ not
      and
    }
    { t #1 #1 substring$
      duplicate$ "-" =
      swap$ duplicate$ "," =
      swap$ "+" =
      or or
        { #1 'multiresult := }
        { t #2 global.max$ substring$ 't := }
      if$
    }
  while$
  multiresult
}

%    \end{macrocode}
% \DescribeFunction{format.pages}
% \changes{v1.2c}{18 Jan 2015}{entry field \texttt{eid} added}
%    \begin{macrocode}
%%FUNCTION {format.pages}
%%{ pages empty$
%%    { "" }
%%    { pages multi.page.check
%%        { bbl.ppages pages n.dashify tie.connect }
%%        { bbl.ppage pages tie.connect }
%%      if$
%%    }
%%  if$
%%}
FUNCTION {format.pages}
{ eid empty$
    {
      pages empty$
        { "" }
        { pages multi.page.check
            { bbl.ppages pages n.dashify tie.connect }
            { bbl.ppage pages tie.connect }
          if$
        }
      if$
    }
    { eid multi.page.check
        { bbl.ppages eid n.dashify tie.connect }
        { bbl.ppage eid tie.connect }
      if$
    }
  if$
}

%    \end{macrocode}
% \DescribeFunction{format.pages.page}
% \changes{v1.2c}{18 Jan 2015}{entry field \texttt{eid} added}
% \changes{v1.2b}{19 Jan 2014}{entry field \texttt{numpages} renamed to \texttt{pagetotal}}
%    \begin{macrocode}
%%FUNCTION {format.pages.page}
%%{ pages empty$
%%    { pagetotal empty$
%%      { "" }
%%      { pagetotal bbl.pages tie.connect }
%%    if$}
%%    { format.pages}
%%  if$
%%}
FUNCTION {format.pages.page}
{ eid empty$
    { pages empty$
        { pagetotal empty$
            { "" }
            { pagetotal bbl.pages tie.connect }
          if$
        }
        { format.pages}
      if$
    }
    { format.pages }
  if$
}

%    \end{macrocode}
% \DescribeFunction{format.vol.num.pages}
%    \begin{macrocode}
FUNCTION {format.vol.num.pages}
{ volume field.or.null
  number empty$
    'skip$
    {
      ", no." number tie.or.space.connect *
      volume empty$
        { "there's a number but no volume in " cite$ * warning$ }
        'skip$
      if$
    }
  if$
  pages empty$
    'skip$
    { duplicate$ empty$
        { pop$ format.pages }
        { ": " * pages n.dashify * }
      if$
    }
  if$
}

%    \end{macrocode}
% \DescribeFunction{format.volume}
%    \begin{macrocode}
FUNCTION {format.volume}
{ volume empty$
    { "" }
    { bbl.vvol volume tie.or.space.connect }
  if$
}

%    \end{macrocode}
% \DescribeFunction{format.number}
%    \begin{macrocode}
FUNCTION {format.number}
{ number empty$
    { "" }
    { bbl.nr number tie.or.space.connect }
  if$
}

%<*debug>
%    \end{macrocode}
% \DescribeFunction{format.chapter.pages}
%    \begin{macrocode}
FUNCTION {format.chapter.pages}
{ chapter empty$
    'format.pages
    { type empty$
        { bbl.chapter }
        { type "l" change.case$ }
      if$
      chapter tie.or.space.connect
      pages empty$
        'skip$
        { ", " * format.pages * }
      if$
    }
  if$
}
%</debug>

%    \end{macrocode}
% \DescribeFunction{empty.misc.check}
%    \begin{macrocode}
FUNCTION {empty.misc.check}
{ author empty$ title empty$ howpublished empty$
  month empty$ year empty$ note empty$
  and and and and and
  key empty$ not and
    { "all relevant fields are empty in " cite$ * warning$ }
    'skip$
  if$
}

%    \end{macrocode}
% \DescribeFunction{format.thesis.type}
%    \begin{macrocode}
%FUNCTION {format.thesis.type}
%{ type empty$
%    'skip$
%    { pop$
%%<!utf8>      bbl.thesis.type "t" change.case$
%%<utf8>      bbl.thesis.type
%    }
%  if$
%}
FUNCTION {format.thesis.type}
{ type empty$
    'skip$
    { pop$
      bbl.thesis.type
    }
  if$
}
%    \end{macrocode}
% \DescribeFunction{format.techrep.type.number}
% Function to format report type and number.
%    \begin{macrocode}
%FUNCTION {format.techrep.type.number}
%{ type empty$
%    { bbl.techreport }
%    'type
%  if$
%  number empty$
%%<!utf8>    { "t" change.case$ }
%%<utf8>    { "" }
%    { number tie.or.space.connect }
%  if$
%}

FUNCTION {format.techreport.type}
{ type empty$
   { bbl.techreport }
   'type
  if$
}

%    \end{macrocode}
%
% \DescribeFunction{author.before}
% Declare the function |author.before| to format list
% of authors in heading of a bibliographic record.
% If the number of authors is 4 or larger, some styles skip
% the list of authors in the beginning of the bibliographic
% record, while other styles always print that list.
% So, we need two version of |author.before|.
%
% First version is used if \file{.bst} style is compiled without option
% |long|. It skips authors if their number is greater than or equal to 4
% or if the author field is empty. Note that GOST requires for a long list of
% authors to be reduced. Hence this first version is preferable.
% Note also that |format.names|  cuts list of names
% to 4 person at most in case if |modern| option is used. and |format.authors|
% just emphasizes |format.names|.
%    \begin{macrocode}
%<*!long>
FUNCTION {author.before}
{
  author empty$
    'skip$
    {author num.names$ #4 <
      {format.authors output
       new.sentence}
      'skip$
    if$}
  if$
}
%</!long>
%    \end{macrocode}
% Second version is used if \file{.bst} style is compiled with the option
% |long|. It skips only if the author field is empty.
%    \begin{macrocode}
%<*long>
FUNCTION {author.before}
{
  author empty$
    'skip$
    { format.authors output
      new.sentence
    }
  if$
}
%</long>

%    \end{macrocode}
% \DescribeFunction{bookauthor.before}
% There are also 2 version of the function |bookauthor.before|.
% Not used anymore!
%    \begin{macrocode}
%%<*!long>
%%FUNCTION {bookauthor.before}
%%{
%%  bookauthor empty$
%%    'skip$
%%    {bookauthor num.names$ #4 <
%%      {format.bookauthors output
%%       new.sentence}
%%      'skip$
%%    if$}
%%  if$
%%}
%%</!long>
%%<*long>
%%FUNCTION {bookauthor.before}
%%{
%%  bookauthor empty$
%%    'skip$
%%    { format.bookauthors output
%%      new.sentence
%%    }
%%  if$
%%}
%%</long>

%    \end{macrocode}
%
% \DescribeFunction{author.after}
% Functions |author.after| and |bookauthor.after| also have by 2 versions.
% They are used to write authors list after the title followed by a slash.
% In modern styles, compiled with option |modern|, the list of authors is
% always cut to at most 4 persons. The cut is performed first by
% |format.names.rev|, which is called by |format.authors.after|. For old styles,
% |author.after| just outputs formatted string whereas for new style it
% skips the string if the number of authors exceeds 3 (and author list is not
% printed before the title).
%    \begin{macrocode}
%<*!modern>
FUNCTION {author.after}
{
  author empty$
    'skip$
    {format.authors.after output
       new.semicolon }
  if$
}
%</!modern>
%<*modern>
%<*!long>
FUNCTION {author.after}
{
  author empty$
    'skip$
    {author num.names$ #3 >
      {format.authors.after output
       new.semicolon }
      'skip$
    if$}
  if$
}
%</!long>
%<*long>
FUNCTION {author.after} { }
%</long>
%</modern>

%    \end{macrocode}
% \DescribeFunction{bookauthor.after}
% This function is used only in |inbook| entry. It always cuts list to 4 persons
% since |format.bookauthors.after| does that.
%    \begin{macrocode}
FUNCTION {bookauthor.after}
{
  bookauthor empty$
    'skip$
    {format.bookauthors.after output
       new.semicolon }
  if$
}

%    \end{macrocode}
% \DescribeFunction{editor.organization.after}
%    \begin{macrocode}
FUNCTION {editor.organization.after}
{
  compiler empty$
    {}
    { format.compiler.after  output
    new.semicolon
    }
  if$
  editor empty$
    {}
    { format.editors.after  output
    new.semicolon
    }
  if$
  organization empty$
    {}
    {organization output
    new.semicolon
    }
  if$
}

%    \end{macrocode}
% \DescribeFunction{format.url}
% \changes{v0.8}{06 Jun 2003}{macro \cs{BibUrl} added}
%    \begin{macrocode}
FUNCTION {format.url}
{ url empty$
    { "" }
    {
%<!(modern|strict)>      "\BibUrl{ " url * "}" *
%<modern|strict>      bbl.url ": \BibUrl{" * url * "}" *
      urldate empty$
        { "" }
        { " (" bbl.urldate * ": " * urldate * ")" * }
      if$ *
    }
  if$
}

%    \end{macrocode}
% \DescribeFunction{output.url}
%    \begin{macrocode}
FUNCTION {output.url}
{
  url empty$
    'skip$
    { format.url output }
  if$
}

%    \end{macrocode}
% \DescribeFunction{format.annote}
% \changes{v0.8}{06 Jun 2003}{macro \cs{BibAnnote} added}
%    \begin{macrocode}
FUNCTION {format.annote}
{ annote empty$
    { "" }
	{ after.sentence 'output.state :=
	  "\BibAnnote{" annote add.period$ * "}" *
	}
  if$
}

%    \end{macrocode}
% \DescribeFunction{format.isbn}
% Do we really need to provide electronic search for ISNB?
%    \begin{macrocode}
FUNCTION {format.isbn}
{
  isbn empty$
    { "" }
    { "ISBN:~\href{http://isbndb.com/search-all.html?kw=" isbn *
      "}{" * isbn * "}" *
    }
  if$
}

%    \end{macrocode}
% \DescribeFunction{add.doi}
% The Digital Object Identifier (DOI) System is for identifying content
% objects in the digital environment. DOI names are assigned to any entity
% for use on digital networks. They are used to provide current information,
% including where they (or information about them) can be found on the
% Internet. Information about a digital object may change over time,
% including where to find it, but its DOI name will not change.
%
% Function |add.doi| embraces last string in stack into hyperlink
% that links it to specified |doi| identificator at http://dx.doi.org/ web-site.
%    \begin{macrocode}
%<*eprint>
FUNCTION {add.doi}
{ duplicate$ empty$
    'skip$
    { doi empty$
        'skip$
        { "\href{http://dx.doi.org/" doi * "}{" * swap$ * "}" * }
      if$
    }
  if$
}
%</eprint>
%    \end{macrocode}
% If \file{.bst} style is compiled without |eprint| option, we just ignore
% |doi| field.
%    \begin{macrocode}
%<*!eprint>
FUNCTION {add.doi} { }
%</!eprint>

%    \end{macrocode}
% \DescribeFunction{add.media}
% New in version 1.2. Adds |media| field if |strict| options is in effect.
% If the |media| field is empty |add.media| prints a value based on the |type|
% field. If the |type| is also empty |add.media| prints equivalent of the
% word |text| in current language.
%    \begin{macrocode}
%<*!strict>
FUNCTION {add.media}
{ duplicate$ empty$
    'skip$
    { media empty$
        'skip$
        { " " * bbl.media enclose.square.brackets * }
      if$
    }
  if$
}
%</!strict>
%<*strict>
FUNCTION {add.media}
{ duplicate$ empty$
    'skip$
    { media empty$
        { " " * bbl.media.text enclose.square.brackets * }
        { " " * bbl.media enclose.square.brackets * }
      if$
    }
  if$
}
%</strict>

%    \end{macrocode}
%
% \subsection{Electronic Publishing Information}
%
% The biblatex package provides three fields for electronic publishing
% information: |eprint|, |eprinttype|, and |eprintclass|. The |eprint| field
% is a verbatim field similar to |doi| which holds the identifier of the item.
% The |eprinttype| field holds the resource name, i. e., the name of the
% site or electronic archive. Optional |eprintclass| field is intended
% for additional information specific to the resource indicated by the
% |eprinttype| field. This could be a section, a path, classification
% information, etc. If the |eprinttype| field is available, the standard styles
% will use it as a literal label. In the following example, they would print
% ``Resource: identifier'' rather than the generic ``eprint: identifier'':
% \begin{verbatim}
% eprint = {identifier},
% eprinttype = {Resource},
% \end{verbatim}
%
% \DescribeFunction{format.eprint}
% The electronic identifier of an online publication.
% This is roughly comparable to a |doi| but specific to a certain archive,
% repository, service, or system.
% Also see eprinttype and eprintclass.
%
% This function should use url. TO BE DONE YET.
% \changes{v1.2}{26 Jan 2012}{entry fields \texttt{eprint}, \texttt{eprintclass}, \texttt{eprinttype}}
%    \begin{macrocode}
%<*eprint>
%FUNCTION {format.eprint}
%{ eprint empty$
%   { "" }
%   { eprintclass empty$
%      { " \href{http://arxiv.org/abs/" eprint * "}" * "{" * eprint * "}" * }
%      { eprinttype empty$
%         { " \href{http://arxiv.org/abs/" eprint * "}" *
%      	   "{" * eprintclass * "/" * eprint * "}" *
%         }
%         { " \href{http://arxiv.org/abs/" eprint * "}" *
%      	   "{" * eprinttype * ":" * eprintclass * "/" * eprint * "}" *
%         }
%      if$}
%   if$}
%if$}

%FUNCTION {format.eprint}
%{ eprint empty$
%   { "" }
%   { eprinttype empty$
%       { "" }
%       { eprinttype "~: " *}
%     if$
%     eprintclass empty$
%       { }
%       { eprintclass * "/" *}
%    if$
%    eprint *
%   }
%  if$
%  url empty$
%    { }
%    { "\href{" url * "}{" * swap$ * "}" *}
%  if$
%}

FUNCTION {format.eprint}
{ eprint empty$
   { "" }
   { eprinttype empty$
       { "" }
       { eprinttype "~: " *}
     if$
     eprintclass empty$
       { }
       { eprintclass * "/" *}
     if$
     url empty$
      { eprint * }
      { "\href{" * url * "}{" * eprint * "}" *}
    if$
   }
  if$
}

FUNCTION {output.eprint.url}
{
  eprint empty$
    { format.url output }
    { format.eprint output }
  if$
}

%</eprint>

%<*!eprint>
FUNCTION {output.eprint.url}
{
    format.url output
}
%</!eprint>

%    \end{macrocode}
%
% Functions added in v1.2f to format patent entry (thanks to Stanislav Kruchinin).
%
% \changes{v1.2f}{12 Jul 2016}{entry field \texttt{patent} added}
% \DescribeFunction{add.number}
%    \begin{macrocode}
FUNCTION {add.number}
{ duplicate$ empty$
    { "" }
    { bbl.nr swap$ tie.or.space.connect }
  if$
}

%    \end{macrocode}
% \DescribeFunction{format.type.number}
%    \begin{macrocode}
FUNCTION {format.type.number}
{
  type empty$
    { "" }
    {
      number empty$
      { "" }
      { type number tie.or.space.connect }
    if$
    }
  if$
}

%    \end{macrocode}
% \DescribeFunction{format.requestdate}
%    \begin{macrocode}
FUNCTION {format.requestdate}
{ requestdate empty$
    { "" }
    { bbl.req requestdate tie.or.space.connect }
  if$
}

%    \end{macrocode}
% \DescribeFunction{format.publicationdate}
%    \begin{macrocode}
FUNCTION {format.publicationdate}
{ publicationdate empty$
    { "" }
    { bbl.publ publicationdate tie.or.space.connect }
  if$
}

%    \end{macrocode}
% \DescribeFunction{format.prioritydate}
%    \begin{macrocode}
FUNCTION {format.prioritydate}
{ prioritydate empty$
    { "" }
    { bbl.priority prioritydate tie.or.space.connect }
  if$
}

%    \end{macrocode}
%
% \subsection{Entry types}
%
% Text below in this section is borrowed from biblatex manual. Not every
% field listed below is actually supported by GOST styles. So description
% below should be considered as a goal or a feature request.
%
% The lists below indicate the fields supported by each entry type.
% Note that the mapping of fields to an entry type is ultimately at the
% discretion of the bibliography style. The lists below therefore serve
% two purposes. They indicate the fields supported by the standard styles
% which ship with this package and they also serve as a model for custom
% styles. Note that the required fields are not strictly required
% in all cases. The fields marked as optional
% are optional in a technical sense. Bibliographical formatting rules
% usually require more than just the required fields. The standard
% styles will generally not perform any formal validity checks, but
% custom styles may do so. Generic fields like abstract and annotation
% or label and shorthand are not included in the lists below because they
% are independent of the entry type.
%
% \subsubsection{Regular Types}
%
% \DescribeFunction{article}
%  An article in a journal, magazine, newspaper, or other periodical which forms a
%  self-contained unit with its own title. The title of the periodical is given in the
%  journaltitle field. If the issue has its own title in addition to the main title of
%  the periodical, it goes in the issuetitle field. Note that editor and related fields
%  refer to the journal while translator and related fields refer to the article.
%
%  Required fields: author, title, journaltitle, year/date.
%
%  Optional fields: translator, annotator, commentator, subtitle, titleaddon,
%  editor, editora, editorb, editorc, journalsubtitle, issuetitle,
%  issuesubtitle, language, origlanguage, series, volume, number, eid, issue,
%  month, pages, version, note, issn, addendum, pubstate, doi, eprint,
%  eprintclass, eprinttype, url, urldate.
%    \begin{macrocode}
FUNCTION {article}
{
  output.bibitem
  author.before
%<natbib>  author format.key output
  format.title add.media "title" output.check
  new.slash
  author.after
  new.dblslash
  journal emphasize add.doi "journal" output.check % new in v.2
  new.block
  format.date "year" output.check
  new.block
  format.volume output
  format.number output
  new.block
  format.pages.page output
  new.block
  note output
  new.sentence
%  format.url output
  output.eprint.url
  format.annote output
  fin.entry
}

%    \end{macrocode}
% \DescribeFunction{book}
% A single-volume book with one or more authors where the authors share credit for
% the work as a whole. In biblatex, this entry type also covers the function
% of the |@inbook| type of traditional BibTeX.
%
% Required fields: author, title, year/date.
%
% Optional fields: editor, editora, editorb, editorc, translator, annotator,
% commentator, introduction, foreword, afterword, subtitle, titleaddon,
% maintitle, mainsubtitle, maintitleaddon, language, origlanguage, volume,
% part, edition, volumes, series, number, note, publisher, location, isbn,
% chapter, pages, pagetotal, addendum, pubstate, doi, eprint, eprintclass,
% eprinttype, url, urldate.
%    \begin{macrocode}
FUNCTION {book}
{
  output.bibitem
  author.before
%<natbib>  author format.key output
  format.btitle add.doi add.media "title" output.check
  new.slash
  author.after
  editor.organization.after
  new.sentence
  format.number.series output
  new.block
  format.edition output
  new.block
  output.address.publisher
  format.date "year" output.check
  new.block
  format.bvolume output
  new.block
  format.pages.page output
  new.block
%<eprint>  format.isbn output
%<eprint>  new.block
  note output
  new.sentence
%  format.url output
  output.eprint.url
  format.annote output
  fin.entry
}

%    \end{macrocode}
% \DescribeFunction{booklet}
% A book-like work without a formal publisher or sponsoring institution. Use the
% field howpublished to supply publishing information in free format, if applicable.
% The field type may be useful as well.
%
% Required fields: author/editor, title, year/date.
%
% Optional fields: subtitle, titleaddon, language, howpublished, type, note,
% location, chapter, pages, pagetotal, addendum, pubstate, doi, eprint,
% eprintclass, eprinttype, url, urldate.
%
%    \begin{macrocode}
FUNCTION {booklet}
{
  output.bibitem
  author.before
%<natbib>  author format.key output
  format.title add.doi add.media "title" output.check
  new.slash
  author.after
  editor.organization.after
  new.block
  howpublished output
  address.or.location output
  format.date "year" output.check
  new.block
  note output
  new.sentence
%  format.url output
  output.eprint.url
  format.annote output
  fin.entry
}

%    \end{macrocode}
% \DescribeFunction{inbook}
% A part of a book which forms a self-contained unit with its own title.
% Note that the profile of this entry type is different from standard BibTeX.
%
% Required fields: author, title, booktitle, year/date.
%
% Optional fields: bookauthor, editor, editora, editorb, editorc, translator,
% annotator, commentator, introduction, foreword, afterword, subtitle,
% titleaddon, maintitle, mainsubtitle, maintitleaddon, booksubtitle,
% booktitleaddon, language, origlanguage, volume, part, edition, volumes,
% series, number, note, publisher, location, isbn, chapter, pages, addendum,
% pubstate, doi, eprint, eprintclass, eprinttype, url, urldate.
%    \begin{macrocode}
FUNCTION {inbook}
{
  output.bibitem
  author.before
%<natbib>  author format.key output
  format.btitle add.doi add.media "title" output.check
  new.slash
  author.after
  new.dblslash
%  bookauthor.before
  booktitle "booktitle" output.check
  new.slash
  bookauthor.after
  editor.organization.after
  new.block
  format.edition output
  new.block
  format.number.series output
  new.sentence
  output.address.publisher
  format.date "year" output.check
  new.block
  format.bvolume output
  new.block
  format.pages.page output
  new.block
%<eprint>  format.isbn output
%<eprint>  new.block
  note output
  new.sentence
%  format.url output
  output.eprint.url
  format.annote output
  fin.entry
}

%    \end{macrocode}
% \DescribeFunction{incollection}
% A contribution to a collection which forms a self-contained unit with a distinct
% author and title. The author refers to the title, the editor to the booktitle,
% i. e., the title of the collection.
%
% Required fields: author, editor, title, booktitle, year/date.
%
% Optional fields: editora, editorb, editorc, translator, annotator,
% commentator, introduction, foreword, afterword, subtitle, titleaddon,
% maintitle, mainsubtitle, maintitleaddon, booksubtitle, booktitleaddon,
% language, origlanguage, volume, part, edition, volumes, series, number,
% note, publisher, location, isbn, chapter, pages, addendum, pubstate, doi,
% eprint, eprintclass, eprinttype, url, urldate.
%    \begin{macrocode}
FUNCTION {incollection}
{
  output.bibitem
  author.before
%<natbib>  author format.key output
  new.sentence
  format.title add.doi add.media "title" output.check
  new.slash
  author.after
  new.dblslash
  booktitle "booktitle" output.check
  new.slash
  editor.organization.after
  new.block
  output.address.publisher
  format.date "year" output.check
  new.block
  format.bvolume output
  format.number.series output
  new.block
  format.pages.page output
  new.block
  note output
  new.sentence
%  format.url output
  output.eprint.url
  format.annote output
  fin.entry
}

%    \end{macrocode}
%
% \DescribeFunction{proceedings}
% A single-volume conference proceedings. This type is very similar to @collection.
% It supports an optional organization field which holds the sponsoring institution.
% The editor is omissible.
%
% Required fields: editor, title, year/date.
%
% Optional fields: subtitle, titleaddon, maintitle, mainsubtitle,
% maintitleaddon, eventtitle, eventdate, venue, language, volume, part,
% volumes, series, number, note, organization, publisher, location, month,
% isbn, chapter, pages, pagetotal, addendum, pubstate, doi, eprint,
% eprintclass, eprinttype, url, urldate.
%    \begin{macrocode}
FUNCTION {proceedings}
{
  output.bibitem
%<natbib>  editor format.key output
  format.btitle add.doi add.media "title" output.check
  new.slash
  editor.organization.after
  new.block
  format.bvolume output
  format.number.series output
  output.address.publisher
  format.date "year" output.check
  new.block
  note output
  new.sentence
%  format.url output
  output.eprint.url
  format.annote output
  fin.entry
}

%    \end{macrocode}
% \DescribeFunction{inproceedings}
% An article in a conference proceedings. This type is similar to @incollection. It
% supports an optional organization field.
%
% Required fields: author, editor, title, booktitle, year/date.
%
% Optional fields: subtitle, titleaddon, maintitle, mainsubtitle,
% maintitleaddon, booksubtitle, booktitleaddon, eventtitle, eventdate,
% venue, language, volume, part, volumes, series, number, note, organization,
% publisher, location, month, isbn, chapter, pages, addendum, pubstate, doi,
% eprint, eprintclass, eprinttype, url, urldate.
% \changes{v0.9}{07 Oct 2003}{bug fix in \texttt{@inproceedings} entry type}
%    \begin{macrocode}
FUNCTION {inproceedings}
{ output.bibitem
  author.before
%<natbib>  author format.key output
  new.sentence
  format.title add.doi add.media "title" output.check
  new.slash
  author.after
  new.dblslash
  booktitle "booktitle" output.check
  new.slash
  editor.organization.after
  new.block
  format.bvolume output
  format.number.series output
  new.block
%  address empty$
%    { publisher output
%      format.date "year" output.check
%    }
%    { address output.nonnull
%      new.colon
%      publisher output
%      format.date "year" output.check
%    }
%  if$
  output.address.publisher
  format.date "year" output.check
  new.block
  format.pages.page output
  new.block
  note output
  new.sentence
%  format.url output
  output.eprint.url
  format.annote output
  fin.entry
}

%    \end{macrocode}
% \DescribeFunction{manual}
% Technical or other documentation, not necessarily in printed form. The author or
% editor is omissible.
%
% Required fields: author/editor, title, year/date.
%
% Optional fields: subtitle, titleaddon, language, edition, type, series,
% number, version, note, organization, publisher, location, isbn, chapter,
% pages, pagetotal, addendum, pubstate, doi, eprint, eprintclass,
% eprinttype, url, urldate.
%    \begin{macrocode}
FUNCTION {manual}
{ output.bibitem
  author empty$
    { organization empty$
        'skip$
        { organization output.nonnull
          address output
        }
      if$
    }
    { format.authors output.nonnull }
  if$
%<natbib>  author format.key output
  new.block
  format.btitle add.doi add.media "title" output.check
  author empty$
    { organization empty$
    {
          address new.block.checka
          address output
        }
        'skip$
      if$
    }
    {
      organization address new.block.checkb
      organization output
      address output
    }
  if$
  format.edition output
  format.date "year" output.check
  new.block
  note output
  new.sentence
%  format.url output
  output.eprint.url
  format.annote output
  fin.entry
}

%    \end{macrocode}
% \DescribeFunction{patent}
%  A patent or patent request. The number or record token is given in the number
%  field. Use the |type| field to specify the type and the |location| field to
%  indicate the scope of the patent, if different from the scope implied by the
%  type. Note that the location field is treated as a key list with this entry
%  type.
%
%  Required fields: author, title, number, year/date.
%
%  Optional fields: holder, subtitle, titleaddon, type, version, location, note,
%  date, month, year, addendum, pubstate, doi, eprint, eprint class, eprint type,
%  url, urldate.
%    \begin{macrocode}
FUNCTION {patent}
{
  output.bibitem
%  title output.nonnull
  title add.media output.nonnull
%  add.blank
%  media enclose.square.brackets output % S.Kruchinin's version
  new.colon
  format.type.number output
  add.blank
  address.or.location output
  new.colon
  ipc output
  new.slash
  format.authors.after "author" output.check
  add.blank
  authorcountry enclose.round.brackets output.nonnull
%<natbib>  author format.key output
  new.semicolon
  holder output.nonnull
  new.semicolon
  credits output.nonnull
  new.block
  requestnumber add.number output
  new.semicolon
  format.requestdate output
  new.semicolon
  format.publicationdate output
  publication output
  new.semicolon
  format.prioritydate output
  prioritynumber output
  prioritycountry enclose.round.brackets output
  new.block
  note output
  new.sentence
  output.url
  format.annote output
  new.block
  pagetotal output
  fin.entry
}

%    \end{macrocode}
% \DescribeFunction{misc}
% A fallback type for entries which do not fit into any other category. Use the field
% howpublished to supply publishing information in free format, if applicable. The
% field type may be useful as well. author, editor, and year are omissible.
%
% Required fields: author/editor, title, year/date.
%    \begin{macrocode}
FUNCTION {misc}
{ output.bibitem
  format.authors output
%<natbib>  author format.key output
  title howpublished new.sentence.checkb
  format.title add.media output
  howpublished new.block.checka
  howpublished output
  new.block
  format.date "year" output.check
  new.block
  note output
  new.sentence
%  format.url output
  output.eprint.url
  format.annote output
  fin.entry
}

%    \end{macrocode}
% \DescribeFunction{unpublished}
% A work with an author and a title which has not been formally published, such
% as a manuscript or the script of a talk. Use the fields howpublished and note to
% supply additional information in free format, if applicable.
%
% Required fields: author, title, year/date.
%
% Optional fields: subtitle, titleaddon, language, howpublished, note,
% location, isbn, date, month, year, addendum, pubstate, url, urldate
%    \begin{macrocode}
FUNCTION {unpublished}
{ output.bibitem
  author.before
%<natbib>  author format.key output
  format.btitle "title" output.check
  new.slash
  author.after
  editor.organization.after
  new.block
  format.date "year" output.check
  new.block
  note "note" output.check
  new.sentence
%  format.url output
  output.eprint.url
  format.annote output
  fin.entry
}

%    \end{macrocode}
% \DescribeFunction{online}
% An online resource. Author, editor, and year are omissible.
% This entry type is intended for sources such as web sites which are intrinsically
% online resources. Note that all entry types support the url field. For example,
% when adding an article from an online journal, it may be preferable to use the
% |@article| type and its url field.
%
% Required fields: author/editor, title, year/date, url.
%
% Optional fields: subtitle, titleaddon, language, version, note,
% organization, date, month, year, addendum, pubstate, urldate.
% \changes{v1.1}{15 Jan 2012}{Entry type \texttt{@online} added}
%    \begin{macrocode}
FUNCTION {online}
{ output.bibitem
  format.authors output
%<natbib>  author format.key output
  title howpublished new.sentence.checkb
  format.title add.doi add.media "title" output.check
%  howpublished new.block.checka
  howpublished new.dblslash.checka
%<!strict>  howpublished output
%<strict>  howpublished enclose.square.brackets output
  editor.organization.after
  new.sentence
  new.block
  output.address.publisher
  format.date output
  new.block
%  format.url output
  output.eprint.url
  new.sentence
  note output
  format.annote output
  fin.entry
}

%    \end{macrocode}
%
% \DescribeFunction{internet}
% \DescribeFunction{www}
% \DescribeFunction{webpage}
% \DescribeFunction{ielectronic}
% New in version 2012.02.15.
%    \begin{macrocode}
FUNCTION {internet}   {online}
FUNCTION {www}        {online}
FUNCTION {webpage}    {online}
FUNCTION {electronic} {online}
%    \end{macrocode}
%
% \DescribeFunction{thesis}
% New in version 2012.02.02.
%
% A thesis written for an educational institution to satisfy the requirements for a
% degree. Use the type field to specify the type of thesis.
%
% Required fields: author, title, type, institution, year/date.
%
% Optional fields: subtitle, titleaddon, language, note, location, month, isbn,
% chapter, pages, pagetotal, addendum, pubstate, doi, eprint, eprintclass,
% eprinttype, url, urldate
%
% \changes{v1.2}{02 Feb 2012}{\texttt{@thesis} entry}
%    \begin{macrocode}
%<*!(modern|strict)>
FUNCTION {thesis}
{ output.bibitem
  format.authors "author" output.check
%<natbib>  author format.key output
  new.sentence
  format.btitle "title" output.check
  new.colon
  bbl.phdthesis format.thesis.type output.nonnull
  new.slash
  %institution "institution" output.check
  institution.or.school "institution/school" output.check
  new.block
  output.address.publisher.date
  new.block
  format.pages.page output
  note output
  new.sentence
  format.url output
  format.annote output
  fin.entry
}
%</!(modern|strict)>
%<*modern|strict>
FUNCTION {thesis}
{ output.bibitem
  format.authors "author" output.check
%<natbib>  author format.key output
  new.sentence
  format.btitle add.doi add.media "title" output.check
  new.colon
  %%type "type" output.check
  bbl.phdthesis format.thesis.type output.nonnull
  new.colon
  %%number output % code of the speciality
  specialitycode.or.number output % code of the speciality, new in v.1.2i
  new.colon
  titleaddon output % date of defence and approvement; new in v.1.2i
  new.slash
  format.authors.after output
  new.semicolon
  %institution "institution" output.check
  institution.or.school "institution/school" output.check
  new.block
  output.address.publisher
  format.date "year" output.check
  new.block
  format.pages.page output
  new.block
  note output
  new.sentence
%  format.url output
  output.eprint.url
  format.annote output
  fin.entry
}
%</modern|strict>

%    \end{macrocode}
%
% \DescribeFunction{report}
% New in version 2012.02.02.
%
% A technical report, research report, or white paper published by a university or
% some other institution. Use the type field to specify the type of report. The sponsoring
% institution goes in the institution field.
%
% Required fields: author, title, type, institution, year/date.
%
% Optional fields: subtitle, titleaddon, language, number, version, note,
% location, month, isrn, chapter, pages, pagetotal, addendum, pubstate, doi,
% eprint, eprintclass, eprinttype, url, urldate.
% \changes{v1.2}{02 Feb 2012}{\texttt{@report} entry}
%    \begin{macrocode}
%FUNCTION {report}
%{
%  output.bibitem
%  author.before
%  new.sentence
%  format.title add.doi add.media "title" output.check
%  new.colon
%%  format.techrep.type.number output.nonnull
%  type "type" output.check
%  new.slash
%  author.after
%  editor.organization.after
%  new.block
%  address output
%  new.colon
%  institution "institution" output.check
%  format.date "year" output.check
%  new.block
%  note output
%  new.block                % v.2
%  format.pages.page output % v.2
%  new.sentence
%%  format.url output
%  output.eprint.url
%  format.annote output
%  fin.entry
%}
FUNCTION {report}
{
  output.bibitem
%  author.before
%  new.sentence
  format.title add.doi add.media "title" output.check
%<natbib>  title format.key output
  new.colon
%  format.techrep.type.number output.nonnull
%  type "type" output.check
%  format.report.type.number "type" output.check
  type "type" output.check
  new.colon
  number output
  new.slash
  %institution "institution" output.check
  institution.or.school "institution/school" output.check
  new.semicolon
  format.chief.after output % from editor field
  new.semicolon
  format.executor.after output % from author field
  new.block
  address.or.location output
  new.colon
  organization output
  format.date "year" output.check
  new.block                % v.2
  format.pages.page output % v.2
  new.sentence % или new.block ?
  output.eprint.url
  new.block
  note output
  format.annote output
  fin.entry
}

%    \end{macrocode}
%
% \subsubsection{Type Aliases}
%
% The entry types listed in this section are provided for backwards compatibility
% with traditional BibTeX styles. These aliases are resolved by BibTeX as the data is
% exported. Bibliography styles will see the entry type the alias points to, not the
% alias name. All unknown entry types are generally exported as @misc.
%
% \DescribeFunction{phdthesis}
% Similar to @thesis except that the type field is optional and defaults to the localized
% term `PhD thesis'. You may still use the type field to override that.
%    \begin{macrocode}
%<*!(modern|strict)>
FUNCTION {phdthesis}
{ output.bibitem
  format.authors "author" output.check
  new.sentence
  format.btitle "title" output.check
  new.colon
  bbl.phdthesis format.thesis.type output.nonnull
  new.slash
  %institution "institution" output.check
  institution.or.school "institution/school" output.check
  new.block
  output.address.publisher.date
  new.block
  format.pages.page output
  note output
  new.sentence
  format.url output
  format.annote output
  fin.entry
}
%</!(modern|strict)>
%<*modern|strict>
FUNCTION {phdthesis}
{ output.bibitem
  format.authors "author" output.check
%<natbib>  author format.key output
  new.sentence
  format.btitle add.doi add.media "title" output.check
  new.colon
  bbl.phdthesis format.thesis.type output.nonnull
  new.colon
  %%number output % code of the speciality
  specialitycode.or.number output % code of the speciality, new in v.1.2i
  new.colon
  titleaddon output % date of defence and approvement; new in v.1.2i
  new.slash
  format.authors.after output
  new.semicolon
  %institution "institution" output.check
  institution.or.school "institution/school" output.check
  new.block
  output.address.publisher
  format.date "year" output.check
  new.block
  format.pages.page output
  new.block
  note output
  new.sentence
%  format.url output
  output.eprint.url
  format.annote output
  fin.entry
}
%</modern|strict>

%    \end{macrocode}
% \DescribeFunction{mastersthesis}
% Similar to |@thesis| except that the type field is optional and defaults to the localized
% term `Master's thesis'. You may still use the type field to override that.
% \changes{v1.2i}{29 Dec 2016}{entry \texttt{@MastersThesis} added again (see v1.1)}
%    \begin{macrocode}
%<*!(modern|strict)>
FUNCTION {mastersthesis}
{ output.bibitem
  format.authors "author" output.check
%<natbib>  author format.key output
  new.sentence
  format.btitle "title" output.check
  new.colon
  bbl.mathesis format.thesis.type output.nonnull
  new.slash
  %institution "institution" output.check
  institution.or.school "institution/school" output.check
  new.block
  output.address.publisher.date
  new.block
  format.pages.page output
  note output
  new.sentence
  format.url output
  format.annote output
  fin.entry
}
%</!(modern|strict)>
%<*modern|strict>
FUNCTION {mastersthesis}
{ output.bibitem
  format.authors "author" output.check
%<natbib>  author format.key output
  new.sentence
  format.btitle add.doi add.media "title" output.check
  new.colon
  bbl.mathesis format.thesis.type output.nonnull
  new.colon
  %%number output % code of the speciality
  specialitycode.or.number output % code of the speciality, new in v.1.2i
  new.colon
  titleaddon output % date of defence and approvement; new in v.1.2i
  new.slash
  format.authors.after output
  new.semicolon
  %institution "institution" output.check
  institution.or.school "institution/school" output.check
  new.block
  output.address.publisher
  format.date "year" output.check
  new.block
  format.pages.page output
  new.block
  note output
  new.sentence
%  format.url output
  output.eprint.url
  format.annote output
  fin.entry
}
%</modern|strict>

%    \end{macrocode}
% \DescribeFunction{docthesis}
% Similar to |@thesis| except that the type field is optional and defaults to the localized
% term `Doctor's of sciences thesis'. You may still use the type field to override that.
% \changes{v1.2i}{02 Jan 2017}{\texttt{@dscithesis} entry renamed to \texttt{@docthesis}}
%    \begin{macrocode}
%<*!(modern|strict)>
FUNCTION {docthesis}
{ output.bibitem
  format.authors "author" output.check
%<natbib>  author format.key output
  new.sentence
  format.btitle "title" output.check
  new.colon
  bbl.docthesis format.thesis.type output.nonnull
  new.slash
  %institution "institution" output.check
  institution.or.school "institution/school" output.check
  new.block
  output.address.publisher.date
  new.block
  format.pages.page output
  note output
  new.sentence
  format.url output
  format.annote output
  fin.entry
}
%</!(modern|strict)>
%<*modern|strict>
FUNCTION {docthesis}
{ output.bibitem
  format.authors "author" output.check
%<natbib>  author format.key output
  new.sentence
  format.btitle add.doi add.media "title" output.check
  new.colon
  bbl.docthesis format.thesis.type output.nonnull
  new.colon
  %%number output % code of the speciality
  specialitycode.or.number output % code of the speciality, new in v.1.2i
  new.colon
  titleaddon output % date of defence and approvement; new in v.1.2i
  new.slash
  format.authors.after output
  new.semicolon
  %institution "institution" output.check
  institution.or.school "institution/school" output.check
  new.block
  output.address.publisher
  format.date "year" output.check
  new.block
  format.pages.page output
  new.block
  note output
  new.sentence
%  format.url output
  output.eprint.url
  format.annote output
  fin.entry
}
%</modern|strict>

%    \end{macrocode}
% \DescribeFunction{conference}
%    \begin{macrocode}
FUNCTION {conference} { inproceedings }

%    \end{macrocode}
%
% \DescribeFunction{techreport}
% TechReport is similar to |@report| except that the |type| field is optional
% and defaults to the localized term `technical report'.
% You may still use the type field to override that.
%    \begin{macrocode}
%FUNCTION {techreport}
%{
%  output.bibitem
%  author.before
%  new.sentence
%  format.title add.doi add.media "title" output.check
%  new.colon
%  format.techrep.type.number output.nonnull
%  new.slash
%  author.after
%  editor.organization.after
%  new.block
%  address output
%  new.colon
%  institution "institution" output.check
%  format.date "year" output.check
%  new.block
%  note output
%  new.block                % v.2
%  format.pages.page output % v.2
%  new.sentence
%%  format.url output
%  output.eprint.url
%  format.annote output
%  fin.entry
%}

FUNCTION {techreport}
{
  output.bibitem
%  author.before
%  new.sentence
  format.title add.doi add.media "title" output.check
%<natbib>  title format.key output
  new.colon
%  format.techrep.type.number output.nonnull
%  type "type" output.check
%  format.report.type.number "type" output.check
%  type output
  format.techreport.type output
  new.colon
  number output
  new.slash
  %institution "institution" output.check
  institution.or.school "institution/school" output.check
  new.semicolon
  format.chief.after output % from editor field
  new.semicolon
  format.executor.after output % from author field
  new.block
  address.or.location output
  new.colon
  organization output
  format.date "year" output.check
  new.block                % v.2
  format.pages.page output % v.2
  new.sentence % или new.block ?
  output.eprint.url
  new.block
  note output
  format.annote output
  fin.entry
}

%    \end{macrocode}
% \DescribeFunction{default.type}
%    \begin{macrocode}
FUNCTION {default.type} { misc }

%    \end{macrocode}
%
% \subsection{Month Abbreviations}
%  Borrowed from \file{merlin.mbs} of package \pkg{custom-bib}. This is done for
%  backward compatibility with standard .bst styles which are designed for English.
%  The string in the definition of any month macro must coincide with that used in
%  |format.month| function in the above.
% \changes{v1.2d}{18 Feb 2016}{\cs{bbljan} etc. macros removed}
%    \begin{macrocode}
MACRO {jan} {"Jan."}
MACRO {feb} {"Feb."}
MACRO {mar} {"Mar."}
MACRO {apr} {"Apr."}
MACRO {may} {"May"}
MACRO {jun} {"Jun."}
MACRO {jul} {"Jul."}
MACRO {aug} {"Aug."}
MACRO {sep} {"Sep."}
MACRO {oct} {"Oct."}
MACRO {nov} {"Nov."}
MACRO {dec} {"Dec."}
%    \end{macrocode}
%
% \subsection{Journal Abbreviations}
%
% \subsubsection{Physics and astronomy}
%
% Borrowed from \file{physjour.mbs} of package \pkg{custom-bib}.
%    \begin{macrocode}
MACRO {aa}{"Astron. \& Astrophys."}
MACRO {aasup}{"Astron. \& Astrophys. Suppl. Ser."}
MACRO {aj} {"Astron. J."}
MACRO {aph} {"Acta Phys."}
MACRO {advp} {"Adv. Phys."}
MACRO {ajp} {"Amer. J. Phys."}
MACRO {ajm} {"Amer. J. Math."}
MACRO {amsci} {"Amer. Sci."}
MACRO {anofd} {"Ann. Fluid Dyn."}
MACRO {am} {"Ann. Math."}
MACRO {ap} {"Ann. Phys. (NY)"}
MACRO {adp} {"Ann. Phys. (Leipzig)"}
MACRO {ao} {"Appl. Opt."}
MACRO {apl} {"Appl. Phys. Lett."}
MACRO {app} {"Astroparticle Phys."}
MACRO {apj} {"Astrophys. J."}
MACRO {apjsup} {"Astrophys. J. Suppl."}
MACRO {apss} {"Astrophys. Space Sci."}
MACRO {araa} {"Ann. Rev. Astron. Astrophys."}
MACRO {baas} {"Bull. Amer. Astron. Soc."}
MACRO {baps} {"Bull. Amer. Phys. Soc."}
MACRO {cmp} {"Comm. Math. Phys."}
MACRO {cpam} {"Commun. Pure Appl. Math."}
MACRO {cppcf} {"Comm. Plasma Phys. \& Controlled Fusion"}
MACRO {cpc} {"Comp. Phys. Comm."}
MACRO {cqg} {"Class. Quant. Grav."}
MACRO {cra} {"C. R. Acad. Sci. A"}
MACRO {fed} {"Fusion Eng. \& Design"}
MACRO {ft} {"Fusion Tech."}
MACRO {grg} {"Gen. Relativ. Gravit."}
MACRO {ieeens} {"IEEE Trans. Nucl. Sci."}
MACRO {ieeeps} {"IEEE Trans. Plasma Sci."}
MACRO {ijimw} {"Interntl. J. Infrared \& Millimeter Waves"}
MACRO {ip} {"Infrared Phys."}
MACRO {irp} {"Infrared Phys."}
MACRO {jap} {"J. Appl. Phys."}
MACRO {jasa} {"J. Acoust. Soc. America"}
MACRO {jcp} {"J. Comp. Phys."}
MACRO {jchp} {"J. Chem. Phys."}
MACRO {jetp} {"Sov. Phys.--JETP"}
MACRO {jfe} {"J. Fusion Energy"}
MACRO {jfm} {"J. Fluid Mech."}
MACRO {jmp} {"J. Math. Phys."}
MACRO {jne} {"J. Nucl. Energy"}
MACRO {jnec} {"J. Nucl. Energy, C: Plasma Phys., Accelerators, Thermonucl. Res."}
MACRO {jnm} {"J. Nucl. Mat."}
MACRO {jpc} {"J. Phys. Chem."}
MACRO {jpp} {"J. Plasma Phys."}
MACRO {jpsj} {"J. Phys. Soc. Japan"}
MACRO {jsi} {"J. Sci. Instrum."}
MACRO {jvst} {"J. Vac. Sci. \& Tech."}
MACRO {nat} {"Nature"}
MACRO {nature} {"Nature"}
MACRO {nedf} {"Nucl. Eng. \& Design/Fusion"}
MACRO {nf} {"Nucl. Fusion"}
MACRO {nim} {"Nucl. Inst. \& Meth."}
MACRO {nimpr} {"Nucl. Inst. \& Meth. in Phys. Res."}
MACRO {np} {"Nucl. Phys."}
MACRO {npb} {"Nucl. Phys. B"}
MACRO {nt/f} {"Nucl. Tech./Fusion"}
MACRO {npbpc} {"Nucl. Phys. B (Proc. Suppl.)"}
MACRO {inc} {"Nuovo Cimento"}
MACRO {nc} {"Nuovo Cimento"}
MACRO {pf} {"Phys. Fluids"}
MACRO {pfa} {"Phys. Fluids A: Fluid Dyn."}
MACRO {pfb} {"Phys. Fluids B: Plasma Phys."}
MACRO {pl} {"Phys. Lett."}
MACRO {pla} {"Phys. Lett. A"}
MACRO {plb} {"Phys. Lett. B"}
MACRO {prep} {"Phys. Rep."}
MACRO {pnas} {"Proc. Nat. Acad. Sci. USA"}
MACRO {pp} {"Phys. Plasmas"}
MACRO {pop} {"Phys. Plasmas"}
MACRO {ppcf} {"Plasma Phys. \& Controlled Fusion"}
MACRO {phitrsl} {"Philos. Trans. Roy. Soc. London"}
MACRO {prl} {"Phys. Rev. Lett."}
MACRO {pr} {"Phys. Rev."}
MACRO {physrev} {"Phys. Rev."}
MACRO {pra} {"Phys. Rev. A"}
MACRO {prb} {"Phys. Rev. B"}
MACRO {prc} {"Phys. Rev. C"}
MACRO {prd} {"Phys. Rev. D"}
MACRO {pre} {"Phys. Rev. E"}
MACRO {ps} {"Phys. Scripta"}
MACRO {procrsl} {"Proc. Roy. Soc. London"}
MACRO {rmp} {"Rev. Mod. Phys."}
MACRO {rsi} {"Rev. Sci. Inst."}
MACRO {science} {"Science"}
MACRO {sciam} {"Sci. Am."}
MACRO {sam} {"Stud. Appl. Math."}
MACRO {st} {"Sky and Telesc."}
%    \end{macrocode}
%
% \subsubsection{Supplementary Journal Names}
%
% Borrowed from \file{suppjour.mbs} of package \pkg{custom-bib}.
%    \begin{macrocode}
MACRO {cjp} {"Czech. J. Phys."}
MACRO {el} {"Europhys. Lett."}
MACRO {en} {"Europhys. News"}
MACRO {fujitsustj} {"FUJITSU Sci. Tech. J."}
MACRO {ieeeed} {"IEEE Trans. Electron Devices"}
MACRO {ieeeim} {"IEEE Trans. Instrum. Meas."}
MACRO {ieeejqe} {"IEEE J. Quantum Electron."}
MACRO {ieeem} {"IEEE Trans. Magn."}
MACRO {ieeeptl} {"IEEE Photonic Technol. Lett."}
MACRO {ieeeuffc} {"IEEE Trans. Ultrason., Ferroelect., Freq. Cont."}
MACRO {jem} {"J. Electron. Mater."}
MACRO {jes} {"J. Electrochem. Soc."}
MACRO {jetplett} {"JETP Lett."}
MACRO {jjap} {"Japan. J. Appl. Phys."}
MACRO {jpha} {"J. Phys. A: Math. Gen."}
MACRO {jphb} {"J. Phys. B: At. Mol. Opt. Phys."}
MACRO {jphbold} {"J. Phys. B: At. Mol. Phys."}
MACRO {jphc} {"J. Phys.: Condens. Matter"}
MACRO {jphcold} {"J. Phys. C: Solid State Phys."}
MACRO {jphd} {"J. Phys. D: Appl. Phys."}
MACRO {jvsta} {"J. Vac. Sci. Technol. A"}
MACRO {jvstb} {"J. Vac. Sci. Technol. B"}
MACRO {me} {"Microelectron. Eng."}
MACRO {necrd} {"NEC Res.{\&} Develop."}
MACRO {pa} {"Physica A"}
MACRO {pb} {"Physica B"}
MACRO {pc} {"Physica C"}
MACRO {pd} {"Physica D"}
MACRO {procieee} {"Proc. IEEE"}
MACRO {procspie} {"Proc. SPIE"}
MACRO {pssa} {"Phys. Stat. Sol. A"}
MACRO {pssb} {"Phys. Stat. Sol. B"}
MACRO {rpp} {"Rep. Progr. Phys."}
MACRO {sm} {"Synthet. Metal"}
MACRO {sost} {"Solid State Technol."}
MACRO {ss} {"Surf. Sci."}
MACRO {ssc} {"Solid State Commun."}
MACRO {sst} {"Semicond. Sci. Technol."}
MACRO {suplatt} {"Superlatt. Microstr."}
MACRO {sust} {"Supercond. Sci. Technol."}
MACRO {znat} {"Z. Naturforsch."}
%    \end{macrocode}
%
% \subsubsection{Optics}
%
% Borrowed from \file{photjour.mbs}.
%    \begin{macrocode}
MACRO {appopt} {"Appl. Opt."}
MACRO {bell}   {"Bell Syst. Tech. J."}
MACRO {ell}    {"Electron. Lett."}
MACRO {jasp}   {"J. Appl. Spectr."}
MACRO {jqe}    {"IEEE J. Quantum Electron."}
MACRO {jlwt}   {"J. Lightwave Technol."}
MACRO {jmo}    {"J. Mod. Opt."}
MACRO {josa}   {"J. Opt. Soc. America"}
MACRO {josaa}  {"J. Opt. Soc. Amer.~A"}
MACRO {josab}  {"J. Opt. Soc. Amer.~B"}
MACRO {jdp}    {"J. Phys. (Paris)"}
MACRO {oc}     {"Opt. Commun."}
MACRO {ol}     {"Opt. Lett."}
MACRO {os}     {"Opt. Spectrosc."}
MACRO {phtl}   {"IEEE Photon. Technol. Lett."}
MACRO {pspie}  {"Proc. Soc. Photo-Opt. Instrum. Eng."}
MACRO {vr}     {"Vision Res."}
MACRO {zph}    {"Z. f. Physik"}
MACRO {zphb}   {"Z. f. Physik~B"}
MACRO {zphd}   {"Z. f. Physik~D"}
%    \end{macrocode}
%
% \subsubsection{Physics of condensed Matter}
%
%    \begin{macrocode}
MACRO {sse} {"Solid-State Electron."}
MACRO {pss} {"Phys. Sol. State"}
MACRO {nl}  {"Nano Lett."}
%    \end{macrocode}
%
% \subsubsection{Soviet and Russian journals}
%  To be extended.
%    \begin{macrocode}
MACRO {sjpp} {"Sov. J. Plasma Phys."}
MACRO {spd}  {"Sov. Phys.--Doklady"}
MACRO {sptp} {"Sov. Phys.--Tech. Phys."}
MACRO       {spu}  {"Sov. Phys.--Uspekhi"}
%<!utf8>MACRO {ufn}  {"\CYRU\CYRF\CYRN"}
%<utf8>MACRO {ufn}  {"УФН"}
MACRO {pu}  {"Phys.--Uspekhi"}
MACRO {sjot} {"Sov. J. Opt. Technol."}
MACRO {sjqe} {"Sov. J. Quantum Electron."}
MACRO {sleb} {"Sov. Phys.--Leb. Inst. Rep."}
MACRO {stph} {"Sov. Phys.--Techn. Phys."}
MACRO {stphl}{"Sov. Techn. Phys. Lett."}
%    \end{macrocode}
%
% \subsection{Main cycle}
%
%    \begin{macrocode}

READ

%    \end{macrocode}
%
% \subsection{Sorting}
%
% Next chunk of code governs sorting reference list by authors' names and
% titles.
%    \begin{macrocode}
%<*sort|natbib>

%    \end{macrocode}
% \DescribeFunction{sortify}
%    \begin{macrocode}
FUNCTION {sortify}
{ purify$
%<!utf8>  "l" change.case$
}
%</sort|natbib>

%    \end{macrocode}
% \DescribeFunction{sort.format.names}
%    \begin{macrocode}
%<*sort>
%% =====================================
%% This version from old Gost package
%%<*!natbib>
FUNCTION {sort.format.names}
{ 's :=
  #1 'nameptr :=
  ""
  s num.names$ 'numnames :=
  numnames 'namesleft :=
    { namesleft #0 > }
    { nameptr #1 >
        { "   " * }
        'skip$
      if$
      s nameptr
      "{vv{ } }{ll{ }}{  f{ }}{  jj{ }}"
      format.name$ 't :=
      nameptr numnames = t "others" = and
        { "et al" * }
        %{ bbl.etal * }
        { t sortify * }
      if$
      nameptr #1 + 'nameptr :=
      namesleft #1 - 'namesleft :=
    }
  while$
}
%%</!natbib>
%% This version from plainnat.bst
%% It ignores second and subsequent authors but include year.
%%<*natbib>
%FUNCTION {sort.format.names}
%{ 's :=
%  #1 'nameptr :=
%  ""
%  s num.names$ 'numnames :=
%  numnames 'namesleft :=
%    { namesleft #0 > }
%    {
%      s nameptr "{vv{ } }{ll{ }}{  ff{ }}{  jj{ }}" format.name$ 't :=
%      nameptr #1 >
%        {
%          "   "  *
%          namesleft #1 = t "others" = and
%            { "zzzzz" * }
%            { numnames #2 > nameptr #2 = and
%                { "zz" * year field.or.null * "   " * }
%                'skip$
%              if$
%              t sortify *
%            }
%          if$
%        }
%        { t sortify * }
%      if$
%      nameptr #1 + 'nameptr :=
%      namesleft #1 - 'namesleft :=
%    }
%  while$
%}
%%</natbib>
%% =====================================

%    \end{macrocode}
% \DescribeFunction{sort.format.title}
%    \begin{macrocode}
FUNCTION {sort.format.title}
{ 't :=
  "A " #2
    "An " #3
      "The " #4 t chop.word % Removes "The " if any
    chop.word               % Removes "An " if any
  chop.word                 % Removes "A " if any
  sortify
  #1 global.max$ substring$
}

%    \end{macrocode}
% \DescribeFunction{author.sort}
%    \begin{macrocode}
%% =====================================
%% This version from old gost package.
%%
%<*!natbib>
FUNCTION {author.sort}
{ author empty$
    { key empty$
        { "to sort, need author or key in " cite$ * warning$
          ""
        }
        { key sortify }
      if$
    }
    {
      author num.names$ #4 <
        {author sort.format.names }
        {title sort.format.title}
      if$
    }
  if$
}
%</!natbib>
%% This version from plainnat.bst
%<*natbib>
FUNCTION {author.sort}
{ author empty$
    { key empty$
        { "to sort, need author or key in " cite$ * warning$
          ""
        }
        { key sortify }
      if$
    }
    { author sort.format.names }
  if$
}
%</natbib>
%% =====================================

%    \end{macrocode}
% \DescribeFunction{author.title.sort}
%    \begin{macrocode}
%<*!natbib|natbib>
FUNCTION {author.title.sort}
{ author empty$
    { title empty$
        { key empty$
            { "to sort, need author, title, or key in " cite$ * warning$
              ""
            }
            { key sortify }
          if$
        }
        { title sort.format.title }
      if$
    }
    {
      author num.names$ #4 <
        {author sort.format.names }
        {title sort.format.title}
      if$
     }  	
  if$
}
%</!natbib|natbib>
%<*natbib|natbib>
FUNCTION {author.editor.sort}
{ author empty$
    { editor empty$
        { key empty$
            { "to sort, need author, editor, or key in " cite$ * warning$
              ""
            }
            { key sortify }
          if$
        }
        { editor sort.format.names }
      if$
    }
    { author sort.format.names }
  if$
}

FUNCTION {author.organization.sort}
{ author empty$
    { organization empty$
        { key empty$
            { "to sort, need author, organization, or key in " cite$ * warning$
              ""
            }
            { key sortify }
          if$
        }
        { "The " #4 organization chop.word sortify }
      if$
    }
    { author sort.format.names }
  if$
}

FUNCTION {editor.organization.sort}
{ editor empty$
    { organization empty$
        { key empty$
            { "to sort, need editor, organization, or key in " cite$ * warning$
              ""
            }
            { key sortify }
          if$
        }
        { "The " #4 organization chop.word sortify }
      if$
    }
    { editor sort.format.names }
  if$
}
%</natbib|natbib>


%    \end{macrocode}
% \DescribeFunction{presort}
% Function to compute |sort.key$|.
% What is the space string \verb*|"   "| for?
%    \begin{macrocode}
%<*!natbib>
FUNCTION {presort}%#1
{
  author.title.sort
  "   "
  *
  year field.or.null sortify
  *
  "   "
  *
  title field.or.null
  sort.format.title
  *
  #1 entry.max$ substring$
  'sort.key$ :=
}
%</!natbib>
%<*natbib>
FUNCTION {presort}%#2
{ calc.label
  label sortify
  %author.title.sort
  "    "
  *
  % ========= plainnat.bst =========
%  type$ "book" =
%  type$ "inbook" =
%  or
%    'author.editor.sort
%    { type$ "proceedings" =
%        'editor.organization.sort
%        { type$ "manual" =
%            'author.organization.sort
%            'author.sort
%          if$
%        }
%      if$
%    }
%  if$
  author.title.sort
  "    "
  *
  year field.or.null sortify
  *
  "    "
  *
  %cite$
  title field.or.null sort.format.title
  *
  #1 entry.max$ substring$
  'sort.label :=
  sort.label *
  % ================================
  #1 entry.max$ substring$
  'sort.key$ :=
}
%</natbib>
%</sort>

%<*!sort>
%<*natbib>
INTEGERS { seq.num }

FUNCTION {init.seq}
{ #0 'seq.num :=}

EXECUTE {init.seq}

FUNCTION {int.to.fix}
{ "000000000" swap$ int.to.str$ *
  #-1 #10 substring$
}

FUNCTION {presort}%#3
{
  calc.label               % computes label
  label sortify            % initiates sort.label
  "    "
  *
  seq.num #1 + 'seq.num := % advance seq.num
  seq.num  int.to.fix      % prepend seq.num with 0s
  'sort.label :=           % set sort.label to seq.num
  sort.label *             % append seq.num to label
  #1 entry.max$ substring$ % cut if too long
  'sort.key$ :=            % set sort.key$
}
%</natbib>
%</!sort>

%<*sort|natbib>
ITERATE {presort}

SORT

%</sort|natbib>

%    \end{macrocode}
%
% \subsection{Bibliography list}
%
% We need to find longest label to put in into the argument of
% the |thebibliography| environment. In case of |natbib| options
% we also need to compute extra suffix for the |year| field if there two or
% more entries for given label (=author/editor/organization) in that year.
%
% Declare global (external) strings used in calculation of the longest label.
%    \begin{macrocode}
%<!natbib>STRINGS { longest.label }
%<natbib>STRINGS { longest.label last.label next.extra }

%<!natbib>INTEGERS { number.label longest.label.width }
%<natbib>INTEGERS { number.label longest.label.width last.extra.num  }

%    \end{macrocode}
%
% \DescribeFunction{initialize.longest.label}
% Initialize those string.
%    \begin{macrocode}
%<*!natbib>
FUNCTION {initialize.longest.label}
{ "" 'longest.label :=
  #1 'number.label :=
  #0 'longest.label.width :=
}
%</!natbib>
%<*natbib>
FUNCTION {initialize.longest.label}
{ "" 'longest.label :=
  #0 int.to.chr$ 'last.label :=
  "" 'next.extra :=
  #0 'longest.label.width :=
  #0 'last.extra.num :=
  #0 'number.label :=
}
%</natbib>

EXECUTE {initialize.longest.label}

%    \end{macrocode}
%
% \DescribeFunction{initialize.longest.label}
% Iterate though the list of entries to compute |label|.
%    \begin{macrocode}
%<*!natbib>
FUNCTION {forward.pass}
{ number.label int.to.str$ 'label :=
  number.label #1 + 'number.label :=
  label width$ longest.label.width >
    { label 'longest.label :=
      label width$ 'longest.label.width :=
    }
    'skip$
  if$
}
%</!natbib>
%<*natbib>
FUNCTION {forward.pass}
{ last.label label =
    { last.extra.num #1 + 'last.extra.num :=
      last.extra.num int.to.chr$ 'extra.label :=
    }
    { "a" chr.to.int$ 'last.extra.num :=
      "" 'extra.label :=
      label 'last.label :=
    }
  if$
  number.label #1 + 'number.label :=
}
%</natbib>

ITERATE {forward.pass}

%    \end{macrocode}
%
% \DescribeFunction{reverse.pass}
% |Natbib| styles require reverse iteration over all entries.
%    \begin{macrocode}
%<*natbib>
FUNCTION {reverse.pass}
{ next.extra "b" =
    { "a" 'extra.label := }
    'skip$
  if$
  extra.label 'next.extra :=
  extra.label
  duplicate$ empty$
    'skip$
    { "{\natexlab{" swap$ * "}}" * }
  if$
  'extra.label :=
  label extra.label * 'label :=
}

REVERSE {reverse.pass}

FUNCTION {bib.sort.order}
{ sort.label  'sort.key$ :=
}

ITERATE {bib.sort.order}

SORT
%</natbib>

%    \end{macrocode}
% \DescribeFunction{begin.bib}
% Within |thebibliography| environment we define few formatting macros
% for user to customize how the reference list is formatted.
% \changes{v1.2a}{31 Aug 2012}{default for \cs{cyrdash} added}
% \changes{v0.8}{06 Jun 2003}{macro \cs{BibAnnote} added}
% \changes{v0.8}{06 Jun 2003}{macro \cs{BibUrl} added}
% \changes{v0.8}{10 Aug 2003}{macro \cs{BibEmph} added}
%    \begin{macrocode}
FUNCTION {begin.bib}
{ "\begin{thebibliography}{"  longest.label  * "}" * write$ newline$
  "\def\selectlanguageifdefined#1{"                  write$ newline$
  "\expandafter\ifx\csname date#1\endcsname\relax"   write$ newline$
%  "\else\language\csname l@#1\endcsname\fi}"         write$ newline$
  "\else\selectlanguage{#1}\fi}"                     write$ newline$
  "\providecommand*{\href}[2]{{\small #2}}"          write$ newline$
  "\providecommand*{\url}[1]{{\small #1}}"           write$ newline$
  "\providecommand*{\BibUrl}[1]{\url{#1}}"           write$ newline$
  "\providecommand{\BibAnnote}[1]{}"                 write$ newline$
  "\providecommand*{\BibEmph}[1]{#1}"                write$ newline$
%<*modern|!modern>
  %"\ProvideTextCommandDefault{\cyrdash}{---}" write$ newline$
  %\DeclareUTFcharacter[\UTFencname]{x2014}{\cyrdash}
  %"\let\cyrdash\textemdash" write$ newline$
  %"\ProvideTextCommandDefault{\cyrdash}{\hbox to.8em{--\hss--}}" write$ newline$
  %"\ProvideTextCommandDefault{\cyrdash}{\textemdash}" write$ newline$
  "\ProvideTextCommandDefault{\cyrdash}{\iflanguage{russian}{\hbox to.8em{--\hss--}}{\textemdash}}" write$ newline$
  %%"\ProvideTextCommandDefault{\cyrdash}{%"  write$ newline$
  %%"  \iflanguage{russian}{\hbox to.8em{--\hss--}}{%" write$ newline$
  %%"    \iflanguage{ukrainian}{\hbox to.8em{--\hss--}}{\textemdash}}}"  write$ newline$
  "\providecommand*{\BibDash}{\ifdim\lastskip>0pt\unskip\nobreak\hskip.2em plus 0.1em\fi" write$ newline$
  "\cyrdash\hskip.2em plus 0.1em\ignorespaces}" write$ newline$
  "\renewcommand{\newblock}{\ignorespaces}" write$ newline$
%</modern|!modern>
%<natbib>  "\providecommand{\natexlab}[1]{#1}" write$ newline$
  preamble$ empty$
    'skip$
    { preamble$ write$ newline$ }
  if$
}


EXECUTE {begin.bib}

EXECUTE {init.state.consts}

ITERATE {call.type$}

%    \end{macrocode}
% \DescribeFunction{end.bib}
%    \begin{macrocode}
FUNCTION {end.bib}
{ newline$
%  "\catcode`\/=11"        write$ newline$
  "\end{thebibliography}" write$ newline$
}

EXECUTE {end.bib}

%</bst>
%    \end{macrocode}
%
% That's all, Folks!
%
% \PrintChanges
% \Finale
%
% \endinput
