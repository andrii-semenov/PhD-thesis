\documentclass[12pt]{vakthesis}

\usepackage[T2A]{fontenc}
\usepackage[utf8]{inputenc}
\usepackage[english,russian,ukrainian]{babel}

\usepackage[intlimits]{amsmath}
\allowdisplaybreaks
\usepackage{amsthm}
\usepackage{amssymb, amsfonts}
%\usepackage{enumerate}
%\usepackage{hyperref}
\usepackage{tabularx}

%\usepackage{multibib}%multiple bibliograpphies
%% The new list's label is "New" and will be titled "The other list".
%% To put cites into this list, use \citeNew.
%\newcites{New}{The other list}

\usepackage[sorting=none, backend=biber]{biblatex} % load the package
%\addbibresource{references.bib} % add a bib-reference file
 

\usepackage{geometry}
\geometry{hmargin={20mm,10mm},vmargin={20mm,20mm}}

\begin{document}

\institution{ОДЕСЬКИЙ НАЦІОНАЛЬНИЙ УНІВЕРСИТЕТ
імені І.І.МЕЧНИКОВА}{Одеса}

\author{СЕМЕНОВ АНДРІЙ КОСТЯНТИНОВИЧ}

\udc{538.956; 537.9; 544.72.05; 544.77}

\title{ЕЛЕКТРОФІЗИЧНІ ВЛАСТИВОСТІ БАГАТОФАЗНИХ\\
ДИСПЕРСНИХ СИСТЕМ}

\speciality[теоретична фізика\\Природничі науки]{01.04.02}[фізико-математичних наук]

\supervisor{Сушко Мирослав Ярославович}
           {кандидат фізико-математичних наук, доцент}

\date{2018}

\maketitle

\chapter*{Аннотація}
mkl

\begin{otherlanguage*}{english}
\begin{center}
    {\normalfont \textbf{ABSTRACT}}
\end{center}
\vskip 40pt

kl
\end{otherlanguage*}

\chapter*{Список публікацій здобувача}
\begin{bibset}[a]{СПИСОК ПУБЛІКАЦІЙ ЗДОБУВАЧА}
\bibliographystyle{unsrt}
\bibliography{xampl-mybib}
\end{bibset}

\tableofcontents

\chapter*{Вступ}
    Де використовуються, навіщо, для чого, чим важливі ефекти\\
    які механізми, які бувають поверхневі шари\\
    які є підходи (числові методи, математичні моделі, симетричні
    та асиметричні моделі), чим вони погані\\
    наш підхід, чим він хороший, як себе показав, чому його
    зручно використовувати\cite{VaZ75, PrB01umc,Pra98,Pie29} %\citeNew{PrB01umc}

\chapter{Модель ядро-оболонка в рамках методу компактних груп неоднорідностей}%%по суті, виклад такий, як у статті про тверді електроліти
\section{Вступ}
    Базові постанови моделі (рисунок, опис моделі) та методу
    (основні формули до явного виду середніх полів через моменти,
    включно)
\section{Моделювання локальних відхилень проникності та розрахунок їх моментів}
\section{Узагальнення на електрично неоднорідні оболонки}
\section{Вибір електродинамічної гомогенізації та основні теоретичні результати}
\section{Висновки}

\chapter{Композити типу діелектрик-провідник}%%grannan,chen,sockov
\section{Вступ}
    що за композити та які ефекти в них присутні,
    чому це важливо, які підходи звичайно використовують
\section{Ефект електричної перколяції}
\subsection{Еффективні критичні індекси перколяції}
\subsection{Порівняння з експериментальними даними провідності}
\section{Поведінка квазістатичної ефективної проникності}%гранан та соцков
\section{??Діелектричні втрати. Ефект Максвела-Вагнера.}%%Соцков
\section{Висновки}

\chapter{Тверді композитні електроліти}
%%siekierski,nan,liang,uvarov
\section{Вступ}
    що це за системи, яки їх особливості, які механізми,
    які теорії існують
\section{Порівняння з числовими розрахунками RRN алгоритму}
\subsection{Відображення неперервної моделі на її дискретний аналог}
\subsection{Порівняння з числовими даними провідності}
\section{Порівняння з реальними експериментальними даними}
\subsection{Ефективна провідність}
\subsection{Ефективна проникність}%уваров
\subsection{??Особливості опису на великих концентраціях}%уваров
\section{Висновки}

\chapter{Полімерні композитні електроліти}%%wieczorek
\section{Вступ}
    що це за системи, яки їх особливості, які механізми,
    які теорії існують
\section{Концентраційна залежність провідності}
\section{Температурна залежність провідності}
\section{Висновки}

\chapter{?????????Емульсії, нанорідини}

\chapter*{Висновки}
    --- розроблена модель універсальна та здатна відновити
    якісно й кількісно експериментальні дані для великої
    кількості дисперсних систем, різноманітного складу\\
    --- критичні індекси перколяції для реальних систем не
    носять універсальний характер, а залежать від області
    концентрацій на яких вони вимірюються\\
    --- поріг перколяції залежить лише від геометричних
    параметрів моделі\\
    --- числовий метод RRN не є відображенням реальної
    системи. щоб так було, треба знаходити коефіцієнти
    квадратури аппрксимованих шарів\\
    --- метод Ханая не є фізично послідовним та не бере
    до уваги багатьох факторів\\
    --- передбачення неоднорідних шарів для полімерних
    електролітів з визначеним якісним профілем провідності
    говорить о якості та здатності розробленої моделі
    до опису систем зі складною структурою поверхневого
    шару\\
    --- ...

\chapter*{Список використаних джерел}

\begin{bibset}{Список використаних джерел}
%\bibliographystyle{gost780}
\bibliographystyle{unsrt}
\bibliography{xampl-thesis}
\end{bibset}

\end{document}
