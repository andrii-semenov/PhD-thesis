% !TeX spellcheck = uk_UA

%У дисертаційний роботі в рамках методу компактних груп неоднорідностей побудована та проаналізована модель невпорядкованої дисперсної системи частинок з морфологією тверде ядро--проникна оболонка, що була далі протестована на числових симуляціях та застосована для обробки експериментальних даних залежностей квазістатичних провідності квазістатичної провідності різних типів твердих композитних та полімерних композитних електролітів та провідності й проникності композитів типу діелектрик--провідник як функцій об'ємної концентрації компонент системи та температури.

Основні висновки з результатів роботи наступні.

\begin{itemize}
	\item Адекватний опис макроскопічних електричних властивостей реальних дисперсноподібних систем вимагає виходу за межі двофазних моделей. Зокрема, він може ефективно здійснюватися в рамках статистичної моделі ефективного електричного відгуку невпорядкованих систем частинок з морфологією тверде ядро--проникна оболонка, побудованої в роботі шляхом узагальнення методу компактних груп на системи провідних частинок.
	
	\item Отримані рівняння для ефективної статичної провідності розглянутих модельних систем підтверджуються результатами порівняння їх розв'язків з даними симуляцій, отриманих методом Random Resistor Network як для електрично однорідних, так і неоднорідних проникних оболонок.  
	
	\item При відповідному виборі одночастинкових профілів провідності оболонок модель кількісно описує експериментальні дані для квазістатичної провідності різних типів твердих композитних та полімерних композитних електролітів. Ці профілі ефективно враховують вплив основних міжфазних та матричних фізико-хімічних механізмів в системі на формування її електричних властивостей та можуть бути використані для аналізу цих механізмів.
	
	\item Також модель кількісно описує поведінку ефективних провідності та діелектричної проникності твердих невпорядкованих композитів типу діелектрик--провідник з проникним міжфазним шаром. Положення порогу електричної перколяції в моделі визначається відносною товщиною оболонки, а значення ефективних критичних індексів  залежать як від геометричних та електричних параметрів компонентів, так і способу обробки експериментальних даних, а тому демонструють широкий спектр значень, спостережуваних на експерименті.
	
	\item Диференціальна схема аналізу ефективних квазістатичних електричних параметрів дисперсних систем застосовна лише для систем з малими різницями діелектричних проникностей компонентів у вузьких концентраційних інтервалах диспергованих компонентів.
\end{itemize}

Таким чином, розроблена модель є новим гнучким інструментом для  електроспектроскопічного аналізу багатофазних дисперсних систем.