% !TeX spellcheck = uk_UA

\paragraph{Актуальність теми}\hfill\par
Актуальність роботи визначається як нагальними практичними 
задачами створення та застосування нових композитних матеріалів 
з бажаними та контрольованими електрофізичними властивостями 
(таких як тверді композитні та полімерні композитні електроліти), 
суттєво відмінними від властивостей природних речовин, так і 
необхідністю побудови і вдосконалення надійних теоретичних 
моделей для кількісного опису та діагностики їх характеристик. 

Робота присвячена побудові та аналізу теоретичної моделі для 
опису найменш дослідженого, але найбільш поширеного типу 
тривимірних невпорядкованих систем, утворених диспергуванням 
частинок наповнювача в несучу матрицю. Ключовими, але
далекими до свого розв'язан\-ня, проблемами при створенні послідовної 
теорії таких систем є врахування різного роду міжфазних 
ефектів (нерегулярність форми частинок; контактний опір; 
утворення оксидних шарів; формування високопровідних областей 
з підвищеною концентрацією дефектів чи іонів; аморфізація 
полімерної матриці тощо), зміна властивостей самої матриці 
(внаслідок неконтрольованого легування, забруднення, зміни 
внутрішньої структури тощо) та послідовне врахування 
багаточастинкових поляризаційних та кореляційних ефектів.

Побудована в дисертаційній роботі аналітична теорія ефективного 
квазістатичного електричного відгуку невпорядкованих систем
частинок з морфологією тверде ядро~-~проникна оболонка є 
багаточастинковою та дозволяє враховувати вплив міжфазних 
та матричних ефектів через моделювання одночастинкового 
електричного профілю комплексної діелектричної проникності 
оболонок. Здобуті основні теоретичні співвідношення між 
ефективною статичною електричною провідністю системи та 
електричними і геометричними параметрами компонентів 
підтверджуються результатами їх порівняння з існуючими даними 
симуляцій методом Random Resistor Network (RRN) та їх спроможністю 
адекватно описувати широкі масиви експериментальних даних 
для ефективної квазістатичної провідності твердих композитних 
і полімерних композитних електролітів, ефективних електричної 
провідності та діелектричної проникності  в околі порогу 
електричної перколяції в системі діелектрик-провідник із 
міжфазним шаром. Теорія також дозволяє показати непослідовність 
та обмеженість поширеної диференціальної схеми для обчислення 
ефективних електричних параметрів гетерогенних систем.



\paragraph{Зв'язок з науковими програмами, планами, темами}\hfill\par
Дисертаційна робота виконувалась на кафедрі теоретичної фізики та 
астрономії Одеського національного університету імені І.~І.~Мечникова, 
а також є складовою частиною досліджень, які проводились
за держбюджетними темами ``Дослідження термодинамічних, критичних та 
кінетичних властивостей рідких металів та їх сплавів'' No 0118U000202, та ``Рівняння стану, термодинамічні та 
кінетичні властивості нанофлюїдів. Дослідження структурування 
нанофлюїдів на основі кореляційної спектроскопії та спектроскопії 
діелектричної проникностi'' No 113U000374.


\paragraph{Мета і задачі дослідження}\hfill\par
\textit{Метою} роботи є побудова теорії ефективних електричних 
властивостей невпорядкованих дисперсних систем частинок з 
морфологією тверде ядро~-~проникна оболонка. У зв’язку з цим 
були поставлені такі \textit{задачі}:

\begin{enumerate}
\item
Розробити теорію електродинамічної гомогенізації невпорядкованих 
систем провідних частинок у рамках методу компактних груп 
(МКГ) \cite{Sushko2007, Sushko2009, SushkoJPD2009, Sushko2017}, 
для чого узагальнити та замкнути МКГ для випадку провідних 
частинок.
\item
Проаналізувати в рамках цієї теорії ефективні електричні 
властивості модельних невпорядкованих систем частинок з морфологією 
тверде ядро~-~проникна оболонка та протестувати теорію шляхом 
порівняння отриманих результатів з даними числових 
симуляцій.
\item
Дослідити застосовність теорії до опису електричних 
властивостей твердих та полімерних композитних 
електролітів.
\item
Дослідити застосовність теорії до опису електричної 
перколяції в дисперсноподібних композитах. 
\item
Виконати в рамках МКГ критичний аналіз диференціальної схеми 
обчислення ефективних електрофізичних параметрів 
гетерогенних систем.
\end{enumerate}

\paragraph{Об'єкт, предмет та методи дослідження}\hfill\par

\textit{Об'єкт дослідження:} невпорядковані дисперсні системи частинок з морфологією тверде ядро~-~проникна оболонка.

\textit{Предмет дослідження:} ефективні електрична провідність та діелектрична проникність.

\textit{Методи дослідження.} У роботі був використаний метод 
компактних груп неоднорідностей~\cite{Sushko2007, Sushko2009, 
SushkoJPD2009, Sushko2017}, який дозволяє врахувати 
багаточастинкові поляризаційні і кореляційні ефекти в 
довгохвильовому наближенні без їх надмірної модельної деталізації.


\paragraph{Наукова новизна отриманих результатів}\hfill\par
В роботі отримано наступні результати:

\begin{itemize}
\item 
В рамках методу компактних груп неоднорідностей побудовано 
внутрішньо замкнену статистичну модель квазістатичного 
електричного відгуку макроскопічно однорідних та ізотропних 
дисперсних систем частинок з морфологією типу тверде 
ядро--проникна оболонка.
\item
Показано адекватність моделі для опису концентраційних 
залежностей статичної провідності, отриманих методом числових 
симуляцій RRN для  модельних систем з електрично однорідними 
та неоднорідними оболонками,  та її суттєві переваги  над 
моделям Максвелла~-~Гарнетта, Бруггемана та Накамури~-~Нана~-~Вєчорика. 
\item
Показано застосовність теорії до кількісного опису 
експериментальних даних з ефективної провідності твердих 
композитних та полімерних композитних електролітів та 
аналізу ролі різних фізико-хімічних механізмів у її формуванні. 
Внески останніх можна ефективно врахувати через модельний профіль 
комплексної діелектричної проникності проникних оболонок. 
\item
Показано застосовність теорії до кількісного опису ефективних 
електричної провідності та діелектричної проникності твердих 
невпорядкованих композитів з міжфазним проникним шаром в околі порогу електричної 
перколяції. Встановлено залежність положення порогу перколяції 
від геометричних параметрів оболонки. Продемонстровано 
залежність ефективних  критичних індексів для таких систем 
від геометричних та електричних параметрів компонентів та 
способу обробки експериментальних даних.
\item
Показано загальну обмеженість 
диференціальної схеми для аналізу ефективних квазістатичних 
електричних параметрів дисперсних систем.
\end{itemize}


\paragraph{Практичне значення отриманих результатів}\hfill\par
Розвинута теорія може розглядатися як новий 
гнучкий інструмент для аналізу та діагностики ефективних 
електрофізичних параметрів широкого кола практично важливих невпорядкованих композитних систем, включаючи
тверді композитні та полімерні композитні електроліти, 
системи типу ізолятор--провідник з міжфазним шаром, колоїди тощо. 
Методи, використані в роботі, можуть бути застосовані
для побудови нових теоретичних моделей ефективних електрофізичних властивостей інших багатофазних систем зі складною мікроструктурою.



\paragraph{Особистий внесок здобувача}\hfill\par
Три статті [1$^*$,~3$^*$,~4$^*$] виконані у співавторстві з науковим керівником. Загальна постановка задач статей [1$^*$,~3$^*$,~4$^*$] та метод компактних груп неоднорідностей належать доц.~Сушку М.Я. При роботі над цими статтями здобувач брав участь в пошуку та аналізі пов'язаних з ними теоретичних матеріалів та експериментальних даних, виконував з науковим керівником паралельні взаємоконтролюючі теоретичні розрахунки та обробки даних симуляцій та експерименту, брав участь в аналізі, інтерпретації результатів та підготовці їх до опублікування. Також здобувачем було виказано ідею про використання крайових умов для встановлення способу замикання процедури гомогенізації, виявлено проблеми, що виникають при відображенні результатів досліджуваної моделі на результати існуючих комп'ютерних симуляцій, запропоновано спосіб відновлення провідності реальної матриці через параметри дальньої частини модельного профілю провідності оболонки.

Постановка задач статей~[2$^*$,~5$^*$] та їх розв'язання належать здобувачеві.



\paragraph{Апробація результатів дисертації}\hfill\par
Результати дисертації доповідалися на семінарах кафедри теоретичної 
фізики, та були представлені автором на наукових
конференціях/школах/семіна\-рах, з яких дванадцять міжнародних:
\begin{enumerate}[leftmargin=1cm] %\textquotedbl
	\item 4-th International Conference ``Statistical Physics: Modern Trends and Applications'', Lviv, Ukraine, 2012.
	\item 25-th International Conference ``Disperse Systems'', Odesa, Ukraine, 2012.
	\item 5-th International Symposium ``Methods and Applications of Computational Chemistry'',  Kharkiv, Ukraine, 2013.
	\item 6-th International Conference ``Physics  of  Liquid  Matter:  Modern Prob\-lems'',  Kyiv, Ukraine, 2014.
	\item 26-th International Conference ``Disperse Systems'',  Odesa, Ukraine, 2014.
	\item 2015 International Young Scientists Forum on Applied Physics,  Dnipropetrovsk, Ukraine, 2015.
	\item 27-th International Conference ``Disperse Systems'', Odesa, Ukraine, 2016.
	\item International conference ``The development of innovation in Engineering, Physical and Mathematical Sciences'', Mykolayiv, Ukraine, 2016.
	\item 8-th International  Conference ``Physics  of  Liquid  Matter: Modern Prob\-lems'', Kyiv, Ukraine, 2018.
	\item 5-th International Conference ``Statistical Physics: Modern Trends and Applications'', Lviv, Ukraine, 2019.
	\item 7-th International Conference ``Nanotechnologies and Nanomaterials'',  Lviv, Ukraine, 2019.
	\item 28-th International Conference ``Disperse Systems'', Odesa, Ukraine, 2019.
\end{enumerate}


\paragraph{Структура та обсяг роботи}\hfill\par
Дисертаційна робота складається зі вступу, п'яти розділів, висновків, списку використаних джерел і додатку. Загальний обсяг дисертації становить 146 сторінки, обсяг основного тексту -- 110 сторінок. Робота містить 8 таблиць, 44 рисунки. Список використаних джерел включає 153 найменування.